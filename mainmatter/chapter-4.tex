\chapter{Stakeholder analysis}
\label{chapter:stakeholders}

In this part of the report, the stakeholders related to Rio Paraná Guazú will be explained. The goal of this stakeholder analysis is to understand the different interests in the region related to sand dredging activities.


\section{Stakeholders}

For the stakeholder analysis, all potentially relevant stakeholders for this project were identified. Some were suggested directly by INA staff, who indicated individuals important for gathering information on the Parana Guazú River. Additional stakeholders were identified through further investigation into relevant contacts. The full list of these stakeholders appears in Table \ref{tab:stakeholders} and will be discussed in the following sections.

\subsubsection{Stakeholders}

\begin{table}[ht]
\centering
\begin{tabularx}{\linewidth}{lX}
\toprule
\textbf{Stakeholder} & \textbf{Role} \\
\midrule
Dredgers & Extraction of sand \\
\midrule
Dredging Companies & Extraction and selling of sand \\
\midrule
Prefectura Naval Argentina & Protects rivers and maritime territory \\
\midrule
Agencia Nacional de Puertos y Navegación & Control of signalling system, dredging, and maintenance \\
\midrule
Ports & Handling and storing of goods \\
\midrule
Fishermen & Local fishermen \\
\midrule
NGOs & Non-profit organizations \\
\midrule
Agua y Saneamientos Argentinos & Delivering water and sewerage services \\
\midrule
Alejo Di Rosio & Filmmaker \\
\bottomrule
\end{tabularx}
\caption{Potential stakeholders}
\label{tab:stakeholders}
\end{table}

\subsubsection{Stakeholders description}

The stakeholders which where given in the overview here above will be shortly described in the following overview. Hereafter, the stakheolders will be described in more detail to get a better understanding of the interests and goals of each specific stakeholder. In Table X an overview of the descriptions is given.

\begin{table}[ht]
\centering
\begin{tabularx}{\linewidth}{lX}
\toprule
\textbf{Stakeholder} & \textbf{Description} \\
\midrule
Dredgers & Extraction of sand \\
\midrule
Dredging Companies & Extraction and selling of sand \\
\midrule
Prefectura Naval Argentina & Protects rivers and maritime territory \\
\midrule
Agencia Nacional de Puertos y Navegación & Control of signalling system, dredging, and maintenance \\
\midrule
Ports & Handling and storing of goods \\
\midrule
Fishermen & Local fishermen \\
\midrule
NGOs & Non-profit organizations \\
\midrule
Agua y Saneamientos Argentinos & Delivering water and sewerage services \\
\midrule
Alejo Di Rosio & Filmmaker \\
\bottomrule
\end{tabularx}
\caption{Potential stakeholders with descriptions}
\label{tab:stakeholders-description}
\end{table}




\subsubsection{Stakeholder interests, problem perception and goal}



\section{Dredgers (LOCAL??)}
The first dominant group of stakeholders are the dredgers.These include the local boats used in the area to extract sand and gravel from the river. There are several different types of dredgers ranging from small independent boats, to groups of small boats that work for the same employer (ARENEROS??), or big extracting ships commonly referred to as 'hoppers'. In the area of interest, the Paraná Guazú from Ibicuy to Brazo Largo, the most common dredgers are: (ARENEROS, INDEPENDENT BOATS). 

Currently there are 2 active boats in the zone of the Rio Paraná Guazú of interest, and (QUITE A LOT MORE IN IBICUY?? FROM INTERVIEW), as mentioned in the previous Chapter. (DREDGING ACTIVITY AANGEVEN WELKE BOTEN, MAPS MET HEEN EN TERUG REIS, DATA , ETC)

Their interest in the Rio Paraná Guazú is to extract the sand in locations of the river that are shallow since they do not possess any technology to mine the sand from deep. Once this water mixed sand is lifted onto the boat, they transport it to the nearest ports, Ibicuy or Brazo Largo. They sell their sand to the highest bidder which can be for (PURPOSES FROM INTERVIEW).


\section{BIG Dredging Companies?}
because they buy the sand and employ the areneros?
The Dredging companies active in the Rio Paraná Guazú are :
(EERST KIJKEN OF YPF, JAn de NUl, etc hierbij kan worden betrokken of niet)


\section{Prefectura Naval Argentina}
The Prefectura Naval Argentina (PNA), is the National Naval Prefecture of the Rio Paraná. Therefore, they are also active in the Paraná Guazú and our area of surveillance. 
It is in their interest to protect the Rio Paraná Guazú from criminal activities. These can 

\section{Agencia Nacional de Puertos y Navegación}
The Agencia Nacional de Puertos y Navegación (ANPYN), or the National Agency of Ports and Navigation, was created on January 6, 2025 by a merger of the Undersecretariat of Ports, Waterways and Merchant Marine and the General Port Administration. Since then, the agency has taken over the tasks of the two former entities and is now responsible for policies concerning ports, waterways and river and maritime transport. As such, keeping the Vía Navegable Troncal (VNT), Argentina's main waterway, navigable for ships is an important task for the agency. To achieve this, they regulate dredging and sand mining contracts and oversee compliance with the relevant regulations .

\section{Ports}
In the researched area there are two strategic port locations that play a key role in the sand extraction.

\subsection{Guazú}
?
\subsection{Ibicuy}
?

\section{Fishermen}
Fischermen is another stakeholder group relevant for this study. The Rio Paraná Guazu is used by a lot of people for its high concentration of fish. 

The artisanal fisheries play an economic role as most of the harvest is sold to middlemen, freezing plants, or in informal markets. Of particular interest are long-range migratory species such as sábalo (Prochilodus lineatus), surubí (Pseudoplatystom corruscans, P. reticulatus), boga (Megaleporinus obtusidens), pacú (Piaractus mesopotamicus), and dorado (Salminus brasiliensis), that support artisanal, recreational, and subsistence fisheries, as observed in other large neotropical rivers, \autocite{assessment of sabalo}, \autocite{fishers' knowledge}

the fischermen in the region of interest are mostly independent. Even though there exists such thing as fischermen associations in the Rio Paraná Guazú, they are not influencial 



\section{NGO's}
even wachten tot ik een nice milieu organisatie vind.

\section{Agua y Saneamientos Argentinos}
Even wachten tot interview met hun, kijken of ze wel relevant zijn


\section{Filmmaker}

When researching for the stakeholders which were relevant for the analysis. We stumble upon an article on a film that was made on the Parana-Paraguay waterway. In this documentary, the focus is on the impact of the trade happening on the Parana-Paraguay waterway. The documentary is made by filmmaker Alejo Di Risio, who was contacted through journalist Matias Avramow for any additional information about stakeholders and business related to the waterway.

