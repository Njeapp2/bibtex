\chapter{Mitigation Strategies}
\label{chap:mitigationstrats}

\section{Nature-based Solutions for bank erosion}

When it comes to Nature-based solutions, the question arises what this definition means. A quite general definition of nature based solutions would be;

\textit{“Nature-based Solutions are actions to protect, conserve, restore, sustainably
use and manage natural or modified terrestrial, freshwater, coastal and marine
ecosystems, which address social, economic and environmental challenges
effectively and adaptively, while simultaneously providing human well-being,
ecosystem services, resilience and biodiversity benefits” (United Nations, 2022,
p. 2)}

Although this definition may give a thought that it only concerns natural and biodiversity increasing ideas this is actually not the case. For example the impact of NBS on the local economies and communities is of an equal importance. When it comes to weighing the different NBS against each other this report will make use of the seven goals of the IUCN which must be achieved as good as possible. The seven goals are presented below;

\begin{figure}[H]
    \centering
    \includegraphics[width=0.50\linewidth]{figures/ThesevenNBSgoals.png}
    \caption{Seven goals for achieving a good NBS}
    \label{fig:placeholder}
\end{figure}

\subsection{Resistance against NBS}

Although NBS are widely known in the scientific world, most people have never heard of these solutions. So, when implementing a solution which can't be described as a classical solution, there is a big change of getting resistance from multiple stakeholders. Especially local communities are skeptical because the solution is less concrete than a classical solution would be. The business case of a NBS must of course also be solid. Without funding of the project, there will never be a change to realize it. That's why it's from great importance to have a solution which is both profitable as explainable to the stakeholders. 

\subsection{Implementing NBS in this project}

As stressed out before in this report, is there a problem with riverbank erosion due to activities on the river. To mitigate or even solve this problem, it is our intend to use a nature based solution. Therefore there are presented multiple possible solutions to create a long-lasting, sustainable riverbank which is made cost effective. These solutions are all graded from one to ten for the seven goals described in section 7.1. The solution which has the highest score will be chosen to mitigate the bank erosion.

One of these solutions is to make a, so said, buffer zone. This means that there is made a swamp area of about 2-8 meter. This gives a change for plants which will hold the soil together. By holding the soil together there will in time be a muddy and organic soil composition, which gives a lot of room for plants and animals to flourish. By doing so, there won't be any erosion on the banks of the river. 


\subsection{Bed and bank protection measures}

belangrijke/interessante info over deltas en wetlands:
The Paraná Delta, the end of the Paraná-Paraguay river wetland system, begins in the city of Diamante in Entre Ríos province. It stretches for 300 kilometres and covers some 2.3 million hectares. Dotted with islands, these wetlands are a source of ecosystem services such as flood and drought buffering, water purification, erosion control and coastal protection, climate regulation, as well as the provision of shelter, feeding and breeding sites for various wildlife species. It also provides resources including fish, foraging, timber, medicine, and materials for construction and clothing.

In recent years, wetlands have become increasingly important for another key reason: their role as allies against climate change. They improve the resilience of communities to its impacts, serve as natural barriers against floods and droughts, and also function as carbon sinks. Despite playing these important roles, these ecosystems remain under great threat from human action – it is estimated that globally, 85 percent of the wetlands that existed three centuries ago have been destroyed or drastically transformed. 

https://dialogue.earth/en/climate/on-the-parana-river-ecological-crisis-is-a-threat-to-its-identity/


\newpage

\section{Structural Solutions for Bank Erosion}
\label{section_8.2}

There are a number of retaining structures that can be used to stabilize the river banks. Possibilities include:
\begin{itemize}
    \item Sheet pile wall\\
    Sheet pile walls are a common retaining structure and consist of vertical barriers made of interlocking sections. They are a lightweight option and can be removed, which makes them reusable across multiple projects. Another advantage is the fact that installation is relatively easy and therefore cheap. However, sheet pile walls also have limitations. In hard soils and soils with boulders or cobbles, installation becomes difficult. Further, installation can disturb nearby areas through sounds and vibrations. These vibrations can even cause settlements to occur \autocite{korffReaderDeepExcavations2023}.

    \item Diaphragm wall\\
    Diaphragm walls are deep, reinforced concrete retaining structures. They provide excellent structural stability and are capable of resisting significant lateral soil and water pressures. One of their key advantages is water tightness, as they effectively prevent groundwater seepage. They are also suitable for a wide range of soil conditions, and offer durability due to the use of reinforced concrete. On the downside, they are costly to build and require significant time and space due to the specialized equipment, skilled labor, and extensive excavation work that is needed \autocite{korffReaderDeepExcavations2023}.
    
    \item Precast concrete wall\\
    Precast concrete walls are constructed by manufacturing structural elements in a factory environment before transporting them to the construction site. This process allows for superior quality control. Moreover, precast construction can significantly speed up project timelines, as elements are produced in large quantities and quickly installed on-site. Precast concrete offers a long service life with minimal maintenance. Drawbacks of precast concrete walls include: the elements are heavy and thus require specialized transportation and installation equipment \autocite{mcneilengineeringAdvantagesDisadvantagesUsing2023}. Further, the production and transport processes have notable environmental impacts, and repairs or replacements can be complex and costly.

    \item Auger pile wall or soldier pile wall\\
    Auger pile walls and soldier pile walls are widely used in construction for retaining slopes. Auger pile walls are formed by drilling and casting concrete in place, while soldier pile walls consist of vertical steel or timber H-piles with horizontal boards or panels placed between them. They are generally cost-effective solutions that generate minimal vibrations, making them suitable for urban areas and sites sensitive to noise or disturbance. Both systems offer flexibility, allowing adjustments to pile placement, size, and depth to suit specific project requirements. However, leakage between adjacent piles is a relevant risk when it comes to these types of walls \autocite{korffReaderDeepExcavations2023}. Maintaining proper overlap between piles is also critical to ensure structural stability and continuity of the wall.
\end{itemize}

In Table \ref{tab:compstruct}, the different structural solutions are summarized and are scored on different relevant criteria.

\begin{table}[H]
\centering
\caption{Comparison of structural solutions}
\resizebox{\textwidth}{!}{%
\begin{tabular}{lcccccccc}
\hline
Method & Installation & Price & Resistance & Versatility & Disturbance & Water tightness & Durability & Sustainability \\
\hline
Sheet pile wall & ++ & + & + & - & - & 0 & + & ++ \\
Diaphragm wall  & -- & -- & ++ & ++ & + & ++ & ++ & - \\
Precast concrete wall & - & - & ++ & 0 & ++ & + & + & -- \\
Auger/Soldier pile wall & + & ++ & 0 & + & ++ & -- & - & 0 \\
\hline
\end{tabular}%
}
\label{tab:compstruct}
\end{table}

As can be seen in Table \ref{tab:compstruct}, pile walls score low on water tightness and durability. The area of interest is located in a delta and hence high groundwater levels are to be expected. Therefore, water tightness must be guaranteed. Since the pile walls don't offer this certainty, this option is not further discussed. The diaphragm wall, on the other hand, offers great water tightness but installation is a far bigger challenge for this method. The benefits that the diaphragm wall offers, great resistance and low disturbance being the most relevant ones, do not outweigh the cons: the large amounts of time, space and budget needed to construct them. The same is true for the precast concrete wall: the heavy elements ask for a specialized and expensive installation procedure. The specialized equipment and experience is possibly not available or expensive, which means the precast concrete walls are not a viable option.

As a structural solution, the sheet pile walls are chosen. These elements score high on ease of installation and sustainability (parts can be removed and reused) and price, resistance and durability are also pros of this method. Disturbance is one of the main concerns related to sheet pile walls, but since the area of interest is in a scarcely populated area, this is not necessarily problematic. Another concern is the low versatility: installation is only possible if soils are not too hard. However, since installation will be executed in a delta with relatively soft soil (see xx), this should not be a major concern for this project.

\subsection{Sheet Pile Wall}
\label{section:sheet_pile_wall}

Sheet pile walls are frequently used in practices for application like excavations, waterfront structures, highway structures, flood protections schemes and bridge abutments. Steel sheet pile walls are mostly used as the high variety of combinations and the various profiles achieve the high moments of resistance while still maintaining the structural requirements of the design. In addition, the engineering advantages are in line with their suitability for water use, favorable ratio of steel cross section and moment of resistance and fast progress on site, which including their functionality and economical benefits is making their use favorable (Handbook).

\subsubsection{Cantilever and anchored}

The types of steel sheet pile walls which are most used in practice are the cantilever walls and anchored walls. Cantilever walls are mostly used as flood walls or earth retaining walls with heights of 3 to 5 meters. They are getting their support reactions from the ground and foundation soils and can be seen in Figure \ref{fig:sheetpiles}. The anchored walls can be used when the heights of cantilever walls are exceeded or when the design should be based on lateral deflections. However, within this design the horizontal distance which will be required for the installation should be considered and and configuration can be seen in Figure \ref{fig:sheetpiles} (EM-1110).

\begin{figure}[H]
    \centering
    \includegraphics[width=0.50\linewidth]{figures/ch8/Anchored-Sheet-pile-wall.png}
    \caption{Cantilever and anchored sheet pile wall}
    \label{fig:sheetpiles}
\end{figure}

% DEZE FIGUUR WELLICHT ZELF MAKEN IN PYTHON. CANTILEVER HEB IK AL. ANCHORED NOG EVEN MAKEN SIMPEL.

\subsubsection{Materials}

Sheet pile walls can be made of multiple materials, like steel, concrete, or wood. In this report, steel sheet pile walls will be used as advantages of this material outweighs the disadvantages and is more suitable than concrete or wood. As concrete will have a long service live, however, it will have high initial costs compared to steel and the installation of concrete walls is more difficult in relation to steel ones. In addition, wood walls, can only be used for short heights and will only be used for temporary structures. Like discussed in Section \ref{section_8.2} the advantages of steel piles and making it therefore the most common material used is the strength, light weight and long service life, in combination with the favorable ratio of cross-section and moment of resistance (EM-1110).

\subsubsection{Sections and interlocks}

For steel sheet pile walls, the sections and interlocks are of importance to make a complete wall which can be used for the application of waterfront structures. Figure \ref{fig:sections_sheetpiles} shows typical steel sections widely used and are known by the names of U and Z sections. In this figure the interlocks, which gives the sections its strength can also be seen. While the interlocks of the U sections are on the neutral axis, the ones for the Z sections are not. The maximum shear stress, can be obtained in the neutral axis, which makes the interlocks should be welded or crimped to obtain the full moment of resistance. When walls are in touch with water, and the walls need to be watertight, materials to fill up the interlocks like plastic compounds or the use of a preformed polyurethane interlock seal can be taken into considerations. 

% NOG EEN BETERE FOTO LATEN ZIEN VAN INTERLOCKS

\begin{figure}[H]
    \centering
    \includegraphics[width=0.50\linewidth]{figures/ch8/u_profile_z_profile.png}
    \caption{U and Z sections}
    \label{fig:sections_sheetpiles}
\end{figure}

\subsubsection{Properties of steel}

As steel will be the material used for the sheet piles the properties of steel, as being a homogeneous material will be touched on. Steel is an elastic material and has a favorable strength to weight ratio while characterizing the a range of 300 $N/mm^{2}$ to 2000 $N/mm^{2}$ for the tensile strength of this material. 

The stress-strain behavior of steel can be seen in Figure \ref{fig:stress_strain_steel}. The range of elasticity is depending on the grade of the steel and the elastic modulus for steel is 210000 $N/mm^{2}$. In Figure \ref{fig:stress_strain_steel}, $f_{y}$ is characterized as the yield strength which is the value where the stress will be constant or drop or go to a strain of 0.2\% when the load is taken away. Furthermore, $f_{u}$, is the tensile strength which is in line with the steel grade (Handbook). The mechanical properties of steel grades which are used for sheet piles are shown in Table \ref{tab:steel_materialproperties}.

% MOOI EN DUIDELIJK STUK SCHRIJVEN OVER DE STRESS-STRAIN RELATIE WAARIN DUIDELIJK WORDT WAT VOOR EEN MATERIAAL STAAL IS EN HOE HET ZICH GEDRAAGD. OOK BACK-UP VAN PAPERS ERIN VERWERKEN.

\begin{figure}[H]
    \centering
    \includegraphics[width=0.50\linewidth]{figures/ch8/stress_strain_steel.png}
    \caption{Stress-strain relation steel}
    \label{fig:stress_strain_steel}
\end{figure}

\begin{table}[ht]
  \centering
  \caption{Mechanical properties by steel grade}
  \label{tab:steel_materialproperties}
  \small
  \setlength{\tabcolsep}{6pt}   % tighten horizontal padding
  \renewcommand{\arraystretch}{1.15}
  \begin{tabularx}{\linewidth}{@{}X
    >{\centering\arraybackslash}p{1.9cm}
    >{\centering\arraybackslash}p{1.9cm}
    >{\centering\arraybackslash}p{2.2cm}@{}}
    \toprule
    \textbf{Steel grade} &
    \makecell{\textbf{Tensile}\\\textbf{strength}\\ $f_{u}\,[\mathrm{N/mm}^{2}]$} &
    \makecell{\textbf{Yield}\\\textbf{strength}\\ $f_{y}\,[\mathrm{N/mm}^{2}]$} &
    \makecell{\textbf{Elongation}\\\textbf{at failure}\\ $\varepsilon_{u}\,[\%]$} \\
    \midrule
    S 240 GP & 340 & 240 & 26 \\
    S 270 GP & 410 & 270 & 24 \\
    S 320 GP & 440 & 320 & 23 \\
    S 355 GP & 480 & 355 & 22 \\
    S 390 GP & 490 & 390 & 20 \\
    S 430 GP & 510 & 430 & 19 \\
    \bottomrule
  \end{tabularx}
\end{table}

\textit{Hot-rolled steel}

Steel sheet piles are made of hot rolled sections, which is the process of heating the steel to temperatures of over 900 $^{\circ}$C, before rolling which allows for the various shape and sizes what is of importance with sheet piles. Hot rolled sections are having slightly rougher surface when comparing it to cold rolled sections (BuyABeam). 

% EVEN HOT ROLLED BESPREKEN. WAT, WAAROM, HOEZO?

\textit{Sustainability}

In Table \ref{tab:env_impacts}, an overview is given for the materials, concrete, steel and timber, in ralation of the CO2 emmisions and the energy consumption. As can be seen from the table, the production and use of steel as a material is not the most sustainable material for the usage for construction. However, as the choice is made of steel as the best option regarding the advantages of the profile and the fast installation of it research has to be done to check the most sustainable options for the steel sheet piles. 

\begin{table}[ht]
  \centering
  \caption{Environmental impacts by material.}
  \label{tab:env_impacts}
  \small
  \setlength{\tabcolsep}{6pt}
  \renewcommand{\arraystretch}{1.15}
  \begin{tabularx}{\linewidth}{@{}lYY@{}}
    \toprule
    \textbf{Material} &
    \textbf{CO$_2$ emissions (kg CO$_2$e per ton)} &
    \textbf{Energy consumption (MJ per ton)} \\
    \midrule
    Concrete & 100 to 200 & 1{,}200 to 1{,}600 \\
    Steel    & 1{,}800 to 2{,}000 & 20{,}000 to 35{,}000 \\
    Timber   & $-600$ to $-1{,}200$ & 500 to 1{,}000 \\
    \bottomrule
  \end{tabularx}
\end{table}

With the knowledge of this sustainable solutions are in need and this is what is in line with the EPD ‘EcoSheetPiles™ Plus’ from ArcelorMittal. These sheet piles are produced on the electric arc furnace which means that in this proces 100\% of renewable electricity is used and 100\% of scrapped steel is being used for the production. In comparison to the blast furnace, the production of CO2 gasses is significantly reduced which can be seen in Figure \ref{fig:eaf_bof}.

\begin{figure}[H]
    \centering
    \includegraphics[width=0.70\linewidth]{figures/ch8/eaf_bof.png}
    \caption{Global warming potential [kg CO2e/t sheet pil] (ArcelorMital)}
    \label{fig:eaf_bof}
\end{figure}



% NOG EVEN GOED UITZOEKEN NAAR WELKE MATERIAL PROPERTIES BELANGRIJK KUNNEN ZIJN. OOK EVEN SUSTAINABILITY IN ACHT NEMEN MET DE GEVONDEN DOCUMENTEN. WELLICHT EEN KOPJE SUSTAINABILITY EN WAT PLOTJES LATEN ZIEN.

\subsubsection{Failure mechanisms}

\begin{figure}[H]
    \centering
    \includegraphics[width=0.70\linewidth]{figures/ch8/failure_mechanisms.png}
    \caption{Failure mechanisms cantilever sheet pile wall}
    \label{fig:failure_mechanisms_sheetpiles}
\end{figure}

% HIER NOG DE FAILURE MECHANISMS BESPREKEN. EVEN KIJKEN WAAROM IK HIER AL ALLEEN DE CANTILEVER LAAT ZIEN. WELLICHT OOK DIE VAN DE ANCHORED LATEN ZIEN OMDAT DE KEUZE PAS VERDER WORDT GEMAAKT.

\subsection{Design of Sheet Pile Wall}

To design the steel sheet pile wall, first the critical location and a problem description should be identified to get a better understanding of the situation. After defining the conditions, a method should be considered to calculate the earth pressure and water pressure to later balance the moments to iteratively define the depth of the sheet pile wall. The sheet pile wall with the retaining height and embedded depth will after be verified based on the structural verifications including bending moments, shear force, and deflection of the wall. 

For the design of the sheet pile wall, the loads acting on the pile wall will be of high importance. In this case, the soil and hydraulic pressure should be defined and to be able to calculate what the acting forces on the structure will be. The soil parameters are defined and the soil profile is taken from Section \ref{par:geology}. With the soil profile and its parameters, the earth pressures will be defined on both sides of the sheet pile wall. For the hydraulic pressure, the water levels are of importance and these are taken from Section XX. 

% HIER EEN MOOI VERHAAL OVER HOE EEN SHEET PILE WALL DESIGN TOT STAND KOMT. 

With the problem description sketched a plan for the design of the calculation of the sheet pile can be set. The plan will be used for the verification of the sheet pile. The steps within the plan are taken from the CUR which is shown in Appendix XX.

% BESPREKEN WAT HET PLAN IS OM EEN DESIGN TE MAKEN EN TE BEREKENEN. CUR GEBRUIKEN.

\subsubsection{Critical locations for design}

The determination of the critical erosion points along the river has been done in Section \ref{section:cirtical_location}. The location which is analyzed here will also be the critical location where the design of the steel sheet pile will be based on. In Figure \ref{fig:critical_location}, the critical point in shown in Aqua Monitor and during the field trip images have been made to verify if the critical location of the Aqua Monitor map was visible. The total length of this location is 215 meters, which is drawn and shown in Table \ref{tab:Surface Lost Camping La Blanqueada in 2022} of Section \ref{section:cirtical_location}.

% DEZE TEKST NOG VERBETEREN

\begin{figure}[H]
    \centering
    \begin{subfigure}[b]{0.45\textwidth}
        \includegraphics[width=\linewidth, height=5cm]{figures/ch8/critical_location_google.png}
        \caption{Critical location Aqua Monitor}
        \label{fig:critical_location_google}
    \end{subfigure}
    \hfill
    \begin{subfigure}[b]{0.45\textwidth}
        \includegraphics[width=\linewidth, height=5cm]{figures/ch8/critical_location.jpeg}
        \caption{Verification critical point field trip}
        \label{fig:critical_location_fieldtrip}
    \end{subfigure}
    \caption{Critical location}
    \label{fig:critical_location}
\end{figure}

\subsubsection{Cross-section of the problem}

For the design of sheet pile wall in this section, the cantilever sheet pile wall will be used. As briefly touched on in Section \ref{section:sheet_pile_wall} the cantilever sheet pile is used for retaining heights up to 5 meters and is getting the support from the ground layers. In Figure \ref{fig:problem_description_sheetpiles}, an sketch of the situation is given. The height which need to be retained in indicated with Z. The embedment depth of the cantilever sheet pile which will need to be determined iterative, is noted as t. In this sketch the ground water table is indicated by the dashed line and abbreviation GWT. The water level of the river is indicated as WL and is in this sketch indicated by WL, which is on the same level as the GWT. 

% MAKKELIJKER UITLEGGEN HOE HET WERKT. WAT ZIJN DE VERSCHILLENDE NIVEAUS? WELKE HOOGTE MOETEN WE BEPALEN. WELKE ZIJN AL BEPAALD. HOE KOMEN WE AAN DE WATERHOOGTE? HOE KOMEN WE AAN DE GROND WATER STAND?

\begin{figure}[H]
    \centering
    \includegraphics[width=0.70\linewidth]{figures/ch8/sketch_profile.png}
    \caption{Problem desciption sheet pile wall}
    \label{fig:problem_description_sheetpiles}
\end{figure}

\begin{itemize}
    \item WL = Water Level
    \item GWT = Ground Water Table
    \item Z = Retained Height
    \item t = Embedding depth
    \item $Z_{1}$ = 
    \item 
\end{itemize}

\subsubsection{Calculation method}

For calculating the embedded depth of a sheet pile wall multiple design methods and models are established worldwide. The most common methods which are applicable for sheet pile walls are the following (Opmaak1):

\begin{itemize}
  \item Bishop
  \item Pressure bar (push up)
  \item PLAXIS (F.E.M)
  \item Blum-sheet pile wall
  \item Spring-supported beam
  \item Terzaghi
  \item Homberg
  \item Horizontal balance
  \item Bakker (PLAXIS)
  \item Piping en heaving
\end{itemize}

Looking at the Eurocode 7, a cantilever sheet pile has to be calculated based on the spring-supported beam method. However, because of the complexity, the sheet pile wall will be calculated based on the the simplified method of BLUM. This method is a simplified representation of the situation, however, the calculations will later be verified by the D-Sheet software of Deltares, in which the spring-supported beam method is integrated.

\textit{Blum-sheet pile wall}

The Blum method is a simplified method in which a graphical approach is used. On the right side of the sheet pile we have the "Active" side and on the left side, under the dredge level, we have the "Passive" side. The simplification step can be seen in Figure \ref{fig:blum} and an additional resultant force is acting on the lowest point of the sheet pile. . In this method, a plastic development of the soil and water table level is assumed to be infinite stiff. To calculate the embeddings depth t, the sum of moments around the lowest point of the sheet pile wall will be set to zero. From the balance of moments the depth can be calculated. However, as in this method no safety factors are taken into account, the depth will have to be multiplied by a factor of 1.2 (Opmaak1)

\begin{figure}[H]
    \centering
    \begin{subfigure}[b]{0.45\textwidth}
        \includegraphics[width=\linewidth, height=5cm]{figures/ch8/blum_1.png}
        \caption{Conventional design}
        \label{fig:conventional_design}
    \end{subfigure}
    \hfill
    \begin{subfigure}[b]{0.45\textwidth}
        \includegraphics[width=\linewidth, height=5cm]{figures/ch8/blum_2.png}
        \caption{Blum simplification}
        \label{fig:blum_simplification}
    \end{subfigure}
    \caption{Blum simplification method}
    \label{fig:blum}
\end{figure}

% DEZE NOG ZELF MAKEN IN PYTHON OOK EEN DUIDELIJK LOPEND VERHAAL MAKEN. BLUM VEREENVOUDIGD. WAAROM MAG DAT? WAAR MOETEN WE OP BLIJVEN LETTEN IN HET VERDERE ONTWERPPROCES?

% \begin{figure}[H]
%     \centering
%     \includegraphics[width=0.70\linewidth]{figures/ch8/blum_methode.png}
%     \caption{Blum method}
%     \label{fig:blum_method}
% \end{figure}

\subsubsection{Geotechnical values and parameters}

To be able to design the sheet pile, the soil layers along with their geotechnical parameters should be known. In Paragraph \ref{par:geology}, the geological background for the area of interest was given and this serves as the basis for deriving soil layers and parameters. The borehole in Figure \ref{fig:borehole} is deemed the most relevant source. The borehole shows a top layer with fine/medium sands. Below, clay/clayey sand can be found, and the bottom layer consists of medium sand. This is in accordance with the geological profile that was provided before in Figure \ref{fig:geolprofile}, which describes a transition from near-surface beach ridges, dunes, beach plains, and delta subaerial facies to deeper open estuaries and marine deposists. The top deposits help declare the presence of sandy deposits at the top of the borehole and the layers of clay/clayey sand below correspond to estuarine deposits. Finally, old marine/fluvial deposits were likely compacted and lead to the layer of medium sand found at the bottom of the borehole.

Because of the resemblance between the local borehole and the geological profile given before, the layering as shown in Figure \ref{fig:borehole} is deemed representative for the whole study area. Based on this layering the relevant parameters can be derived, the result is shown in Table \ref{tab:soil_layers}.

\begin{table}[ht]
  \centering
  \caption{Soil stratigraphy and geotechnical properties.}
  \label{tab:soil_layers}
  \small % slightly smaller text to help fitting
  \setlength{\tabcolsep}{6pt}   % tighten horizontal padding
  \renewcommand{\arraystretch}{1.15}
  \begin{tabularx}{\linewidth}{@{}p{1.6cm}p{1.6cm}l*{5}{Y}@{}}
    \toprule
    \textbf{Start layer (m)} &
    \textbf{End layer (m)} &
    \textbf{Soil type} &
    $\boldsymbol{\gamma_d}\,[\mathrm{kN/m}^3]$ &
    $\boldsymbol{\gamma_{\!sat}}\,[\mathrm{kN/m}^3]$ &
    $\boldsymbol{\varphi'}\,[{}^\circ]$ &
    $\boldsymbol{c'}\,[\mathrm{kPa}]$ &
    $\boldsymbol{c_u}\,[\mathrm{kPa}]$ \\
    \midrule
     0.0    &  2.0    & Fill (Topsoil/Agricultural) & 12 & 12 & 15.0 & 2.5 & 20 \\
     2.0    &  7.0    & Fine/medium sand            & 17 & 19 & 30.0 & 0.0 & \textemdash \\
     7.0    & 10.0    & Clay                         & 14 & 14 & 17.5 & 0.0 & 25 \\
    10.0    & 15.0    & Clayey sand                  & 18 & 20 & 25.0 & 0.0 & \textemdash \\
    15.0    & 16.0    & Clay                         & 14 & 14 & 17.5 & 0.0 & 25 \\
    16.0    & 17.5  & Clayey sand                  & 18 & 20 & 25.0 & 0.0 & \textemdash \\
    17.5  & 32.0    & Medium sand                  & 18 & 20 & 32.5 & 0.0 & \textemdash \\
    \bottomrule
  \end{tabularx}
\end{table}

% TABEL VERBETEREN NAAR DE JUISTE FORMAT. EVEN KIJKEN NAAR DE TABEL UIT PYTHON

The parameters in Table \ref{tab:soil_layers} were derived from the Eurocode \autocite{stichtingkoninklijknederlandsnormalisatieinstituutNederlandseNormNEN2025}. Because of limited knowledge on soil characteristics, conservative estimates were made based on this code. In the borehole, no explicit information is given on the top fill layer. Therefore, conservative parameters were assumed based on typical values for organic topsoil.

\textit{Soil profile in cross-section}

As derived from the borehole and the overview in Table \ref{tab:soil_layers}, the soil profile with the specific layers is added to the cross-section of the problem. In Figure \ref{fig:soil_profile_cross_section} it can be seen that the geotechnical properties of each layers is displayed. 

\begin{figure}[H]
    \centering
    \includegraphics[width=0.70\linewidth]{figures/ch8/soil_profile_cross_section.png}
    \caption{Soil profile in the cross-section}
    \label{fig:soil_profile_cross_section}
\end{figure}

\subsubsection{Hydraulic values and parameters}

In addition to the geotechnical values and parameters, the design of the cantilever sheet pile wall is also based on the hydraulic parameters and values. As if a sheet pile wall is located in opens water or when having groundwater the hydrostatic pressure will be acting on both sides of the sheet pile wall. The water level of the river and the ground water can be of difference, and when this is the case an excess of hydrostatic pressure will be acting on the sheet pile. 

The hydrostatic pressure w is called the pore water pressure. In unconfined groundwater, the pore hydrostatic pressure at a depth z is calculated with the use of Equation \ref{eq:pore_water_pressure}.

\begin{equation}
    w = z \cdot \gamma_{w}
    \label{eq:pore_water_pressure}
\end{equation}

When there is an excess of hydrostatic pressure, which means that the water levels on both side of the sheet pile wall are not on the same level, the excess hydrostatic pressure will occur. The excess hydrostatic pressure can be seen in Figure \ref{fig:excess_pressure} and is calculated with Equation \ref{eq:excess_water_pressure}.

\begin{equation}
    w_{u}(z) = w_{r}(z) - w_{l}(z) = h_{r}(z) \cdot \gamma_{w} - h_{l}(z) \cdot \gamma_{w}
    \label{eq:excess_water_pressure}
\end{equation}

\begin{figure}[H]
    \centering
    \begin{subfigure}[b]{0.45\textwidth}
        \includegraphics[width=\linewidth, height=5cm]{figures/ch8/water_pressure.png}
        \caption{Hydrostatic pressure}
        \label{fig:hydrostatic_pressure}
    \end{subfigure}
    \hfill
    \begin{subfigure}[b]{0.45\textwidth}
        \includegraphics[width=\linewidth, height=5cm]{figures/ch8/excess_water_pressure.png}
        \caption{Excess hydrostatic pressure}
        \label{fig:excess_hydrostatic_pressure}
    \end{subfigure}
    \caption{Excess hydrostatic pressure}
    \label{fig:excess_pressure}
\end{figure}

% DEZE PLOT NOG VERBETEREN IN DE CODE EN DUIDELIJKER MAKEN EN DE WAARDES WEGHALEN

% \begin{figure}[H]
%     \centering
%     \includegraphics[width=0.70\linewidth]{figures/ch8/soil_profile_cross_section.png}
%     \caption{Soil profile in the cross-section}
%     \label{fig:soil_profile_cross_section}
% \end{figure}

\subsubsection{Safety concept}

% HIER BESPREKEN VAN DE VEILIGHEIDSFACTOREN VAN ZOWEL DE KRACHTEN ALS DE GRONDPARAMETERS. OF DIT PAS DOEN BIJ DE VOOR DE BERKENING ZELF.

\subsubsection{Calculation of minimal embedding depth}

To calculate the embedding depth of the sheet pile. The moments around the lowest point of the sheet pile should be set to zero. To make up the balance of moments, firstly the forces and pressures acting on the sheet pile should be defined. The horizontal pressure acting on the sheet pile are based on the earth pressure by the soil and the hydrostatic pressure by the water. Firstly, the earth pressures will be calculated where after the hydrostatic pressure will be calculated, which will later be described as forces.

The depth of the embedding can be calculated in multiple ways. One way is to give the depth an unknown parameter t and leave this unknown expressed in the pressure and forces, which later give an explicit formula where the depth in an unknown. This can therefore be calculated in the sum of moments with taking roots. However, another approach is to begin with an estimated depth and calculate what the sum of moments around the lowest point is. After an iterative solution can be used to make sure the depth is increasing while optimizing the moment balance to zero at the lowest part of the sheet pile. In this calculation for the depth of the sheet pile the second method is taken and a first estimation of the embeddings depth is taken. The parameters with the given values can be seen in Figure XX.

\textit{Effective stress}

The horizontal pressure of the soil can be calculated based on the vertical effective stress. The effective stress in the soil can be calculated with Equation \ref{eq:effective_stress}. In which the effective stress is based on the weight of the soil and the unit weight $\gamma$, of each layer. If the soil layer is below the ground water table the saturated unit weight should be deducted from the unit weight, $\gamma - \gamma_{w}$. 

\begin{equation}
    \sigma^{'}_{v} = \gamma \cdot z
    \label{eq:effective_stress}
\end{equation}

In Figure XX the effective stress of both sides of the sheet pile can be seen. In Table XX, the related values of the effective stress are displayed.



\textit{Earth pressure}

Uitleg over hoe de active en passive earth pressure berekend wordt met een afbeelding en een tabel met de belangrijkste waarden. Uitleg dat alles onder de dredging line een waarden zal komen die uit een onbekende D zal bestaan. 

\textit{Hydrostatic pressure}

Uitleg over hoe de hydrostatic water pressure berekend wordt met een afbeelding van de passive en active side en een tabel met de belangrijkste waarden. Uitleg dat alles onder de dredging line een waarden zal komen die uit een onbekende D zal bestaan.

\textit{Loads}

Hier komt een beschrijving van de verschillende krachten die op de damwand komen te staan. Denk hierbij aan de earth pressure, water pressure en de golf energie force.

\textit{Balance of Forces}

Vanuit de active en passive earth pressure de krachten kunnen bepalen doormiddel van vermenigvuldiging met de hoogte van de laag. Alle krachten laten zien in een tabel en ook een afbeelding van de krachten per driehoek en vierkant. Dit in de tabel mooi laten zien. Ook de balanssom van krachten beschrijven.

\textit{Balance of Moments}

De momentensom om het laagste punt van de sheet pile gelijkstellen aan 0. Dit laten zien in een afbeelding en de uiteindelijke vergelijking die bestaat ui waardes en een onbekende D. Hieruit de wortel en d bepalen die de sheet pile zal moeten hebben. Ook een plaatje van de momenten lijn over de gehele damwand laten zien.

\subsection{Structural Verification}

\subsubsection{ULS and SLS}

\subsubsection{Normal Force}

\subsubsection{Moments}

\subsubsection{Deflection}

\subsection{Conclusion}

\newpage

\section{Nature-based Solutions for dry sand mining}
Meer overheidsingrijpen nodig? Zie interview burgemeester: The judiciary forced the government to get involved: do controls, plan and regulate. Now there’s
a limit to extract, approx. 2 – 3 m. Before, there wasn’t.
En sed de arena report, overheid niet proactief genoeg

%https://www.unep.org/resources/report/sand-and-sustainability-10-strategic-recommendations-avert-crisis
%https://aapepyg.com/2022/11/23/fracking_entre_rios/#_edn4

While significant strides are taken to implement
nature-based solutions (NbS) against climate change
challenges, it is important to note that NbS requires large
volume of sand. Additionally, NbS may take time to have
the desired effect. Thus in some cases, grey structures
(i.e., concrete) may be necessary in the mean time to
address challenges in the short- to medium-term. In the
context of extreme heat and its impacts on cities, NbS
will also be instrumental to promote building designs and
materials that require neither concrete infrastructure nor
sand (UNEP 2021b). Climate change induced pressures,
such as temperature stress and precipitation, will also
speed up the degradation of existing infrastructure and
the need for upgrading or replacement
https://www.unep.org/resources/report/sand-and-sustainability-10-strategic-recommendations-avert-crisis

Op basis van uitgebreid, door deskundigen beoordeeld bewijsmateriaal concludeerde het Compendium van wetenschappelijke, medische en media-bevindingen die de risico's en schade van fracking aantonen, dat het niet mogelijk is om de techniek van de winning van onconventionele koolwaterstoffen 17Sed de Arena 2023 I Valeria Foglia en de daarmee samenhangende winningsactiviteiten niet mogelijk is zonder dat dit een bedreiging vormt voor de menselijke gezondheid, de lucht, het water, de economische vitaliteit op lange termijn, de biodiversiteit en de seismische en klimatologische stabiliteit. SED DE ARENA

This section seeks to propose and evaluate Nature-Based Solutions (NBS) that can effectively address the environmental and socio-economic consequences of dry sand mining. The analysis focuses on developing strategies that harness natural processes to restore degraded ecosystems, enhance landscape resilience, and promote sustainable resource management. In order to provide a comprehensive understanding, the potential advantages and limitations of each proposed NBS will be critically examined. Furthermore, this section will delineate the intended implementation framework, outlining the methodological approach and practical steps required to translate these solutions into effective, context-specific interventions.

\subsection{Floodplains}

Floodplains are low-lying areas around rivers that naturally flood during periods of high water discharge. When managed and restored appropriately, these zones play a crucial role in maintaining riverine ecosystem functions and mitigating the adverse impacts of human activities such as sand mining and land degradation.

In the context of dry sand mining, floodplains could be a Nature-Based Solution to counteract the degradation of terrestrial and hydrological systems resulting from excessive sand extraction. Dry sand mining frequently occurs in former floodplain areas or adjacent uplands, where the removal of surface materials disrupts soil structure, alters drainage patterns, and reduces the area’s natural capacity to retain water. Restoring and reactivating floodplains in such landscapes helps to re-establish the natural hydrological connectivity between surface and subsurface systems. This process promotes groundwater recharge, initiate soil sedimentation and mitigates erosion caused by wind and surface runoff. Moreover, re-vegetated floodplains provide new habitats for native species, contributing to the overall ecological recovery of mined areas. As such, floodplain restoration in dry sand mining zones supports landscape resilience, reduces the long-term environmental footprint of extraction activities, and facilitates a more sustainable post-mining land use.

Restored floodplains offer significant opportunities for local community engagement, livelihood diversification, and socio-economic development. Once stabilized and revegetated, these areas can be utilized for sustainable land uses such as flood-resilient agriculture, agroforestry, and controlled grazing, which maintain ecological functions while providing income for local populations. Additionally, floodplains can serve as sites for eco-tourism, recreation, and environmental education, fostering a stronger connection between communities and their natural surroundings. The enhancement of biodiversity and landscape aesthetics further increases the cultural and recreational value of these areas. Moreover, the restored floodplain’s role in improving water retention and soil fertility can directly support local food and water security, especially in regions affected by the environmental degradation of dry sand mining. Through community-based stewardship, floodplain restoration can therefore become a catalyst for sustainable rural development and long-term environmental resilience.

\subsubsection{Implementation in Ibicuy area}

\subsection{Vegetated buffer zones}
Vegetated buffer zones are strips of land planted with dense vegetation, such as grasses, shrubs, and trees, established between areas of active land use (dry mining sites) and surrounding environments such as settlements, agricultural fields and nature. In the context of dry sand mining, these buffers function as a natural barrier that mitigates the spread of dust, noise, and pollutants generated by extraction activities.

From an ecological perspective, vegetated buffer zones play a vital role in stabilizing soil, reducing wind speed, and capturing airborne particles, thereby improving local air quality and minimizing off-site impacts. The root systems of the plants anchor the sandy substrate, preventing erosion and surface runoff, while the vegetation canopy traps dust and enhances microclimatic conditions by increasing humidity and reducing temperature fluctuations.

In addition to their environmental benefits, buffer zones contribute to biodiversity enhancement by creating transitional habitats that support a variety of species and small fauna. They also serve as visual and acoustic screens, reducing noise and improving the aesthetic quality of the landscape. When designed with native and drought-tolerant species, vegetated buffers require minimal maintenance and can thrive in the disturbed conditions typical of mining landscapes.

Moreover, the establishment of buffer zones can provide socio-economic advantages for local communities through community-based planting initiatives. Overall, vegetated buffer zones represent a cost-effective and multifunctional Nature-Based Solution that simultaneously addresses environmental degradation, air quality concerns, and landscape restoration in areas affected by dry sand mining.
