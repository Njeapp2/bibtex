\chapter{Discussion}
1: Summarize key findings
Summarize key results
Short, only main question

\section{Interpretation}
2: Interpretation of results
Results per theme/question?
Interpret results:
- Identify correlations, patterns, and relationships in data
- Did results meet expectations or hypotheses
- Contextualize findings within previous research and theory
- Explain unexpected results and evaluate their significance
- Consider possible alternative explanations and make an argument for your position

Stakeholders:
Stakeholders identified, with goals and interests
Key players: ports, dredgers, ANPYN
Interview results: two ships active today, river sand for construction, low demand so for fracking likely
Effects: erosion of 30 meters from dredging and cargo
Dry sand mining: 350 trucks with 9000 tons of sand daily, road conditions, supply fracking industry, transported to Anelo
Stakeholders updated: campings more power, dredgers less, ports less interest and power, municipalities introduced. Oppositon increased for landowners and campings

Sand extraction:
AIS data shows 2 ships, 3 trips per day
Volumes calculated
Interview results: two ships active today
Dry sand mining: fracking more and more important, upward trend seen as well 
This indicates that the upgoing trend has since persisted.
It seems therefore clear that fracking practices are the driving force behind the increasing demand of sand. This is also the conclusion that other reports reach \autocite{secretariadepoliticamineraArenasParaFracking2019} \autocite{fogliaSedArena2023}.
As mentioned in chapter \ref{chapter:stakeholders}, the mayor of Ibicuy described the following scale during the conducted interview: 350 trucks that transport 9,000 tons of sand each day. With a total of 260 working days per year, this amounts to more than 2.3 million tons of sand transported from Ibicuy in 2025. In 2022: 1.250.000 tons in Ibicuy, so further increase can be seen.
Multiple stakeholders, such as the mayor of Ibicuy and the mine manager, explained that mined sand gets transported by trucks to Añelo, a town in Neuquén that forms the heart of the Vaca Muerta fracking activities. This view is confirmed by various reports \autocite{cauceArenasParaFracking2022} \autocite{secretariadepoliticamineraArenasParaFracking2019}.
Geology: borehole, characteristics sand
Consequences: natural habitat, social, economic effects, stakeholders
The most frequently voiced concerns of stakeholders is the poor conditions of roads, especially the provincial RP45 that forms the entrance to the town, caused by intense heavy truck traffic. These comments were confirmed during the field trip, since many potholes were observed along the entire route.

Hydrodynamical and sedimentary analysis:
xx

Delft3D Model:
xx

Mitigation strategiesL

3: Implication
Do results support or challenge existing theories and literature? 
If they support existing theories, what new information do they contribute?
If they challenge existing theories, why do you think that is?
Are there any practical implications?

4: Limitations
provide an accurate picture of what can and cannot be concluded from your study.
Limitations might be due to research design, methodological choices, or unanticipated obstacles that emerged.
Reiterate why the results are valid for answering research question.