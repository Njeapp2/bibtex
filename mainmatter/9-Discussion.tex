\chapter{Discussion}
%1: Summarize key findings Summarize key results Short, only main question

\section{Sand extraction volumes, purpose and the demand for sand}
Interviews with local stakeholders reveal that dredging activity in recent years has stayed concstant in certain parts of the study area. In the Paraná Guazú, two dredging companies are operational, as both AIS data and interview data show. Stakeholders did not speak of a trend in these areas. In Puerto Ibicuy, river sand extraction taxes were increased significantly, leading to the discontinuation of almost all dredging activities on the Paraná Ibicuy. Historic AIS data was not available, but present-day AIS data as well as fieldwork observations confirm that no dredging activities on the Paraná Ibicuy take place anymore. This finding directly contradicts the initial hypothesis suggesting a potential increase in river sand mining. Instead, it highlights the influence of local governance as a powerful regulator of extraction activities.

In contrast, dry sand mining has expanded dramatically. Historic data shows the increasing importance of fracking gas and oil from Vaca Muerta since its discovery in 2011. This trend goes hand-in-hand with the trend in mined sand masses in Argentina, increasing from around 500,000 tons in 2011 to 3.5 million tons in 2020. Although sand is also used by the glass and ceramics and the exact demand from different sectors is unknown, the upgoing trend in both data sets is a clear indicator that mined sand is used more and more for fracking. This view is supported by stakeholder interviews and previous research \autocite{fogliaSedArena2023} \autocite{secretariadepoliticamineraArenasParaFracking2019}. As mentioned in chapter \ref{chapter:stakeholders}, the mayor of Ibicuy described the following scale during the conducted interview: 350 trucks that transport 9,000 tons of sand each day. With a total of 260 working days per year, this amounts to more than 2.3 million tons of sand transported from Ibicuy in 2025. In 2022, the mined mass was 1.250.000 tons in Ibicuy \autocite{fogliaSedArena2023}. This indicates a further growth of the sand mining industry in Ibicuy in recent years, the bulk of which is driven by the fracking practices in the South of the country. This supports initial expectations, although stakeholder interviews and river extraction data indicate that for the area of interest, the increase is due to dry sand mining and not due to river mining. 

However, as becomes clear from the oil and gas production trends in Argentina, fracking forms a greater and greater share of the national production. Fracking in the Neuquén basin is viewed by some as critical to the development of Argentina and the Government of Argentina still views the oil and gas sector as a crucial part of its economy, by driving exports as well as generating foreign currency and investment. It also becomes clear that, considering the recoverable volumes present in the reserves (16 billion barrels of oil and 8722 billion cubic meters of gas), even more resources could still be extracted. The upgoing trend in fracking is expected to increase further and it is estimated that the number of fractures in 2025 will increase by 25\% as compared to 2024 \autocite{barnedaFrackingVacaMuerta2025}. A further increase in fracking will likely lead to more sand extraction from the study area and can ultimately lead to increased sand extraction from the river. After all, stakeholders indicated that river sand can be used for fracking and the dredger mentioned that selling to the fracking industry could be likely.

Fracking sand, also known as silica sand, must meet strict physical and chemical standards to be suitable for use in hydraulic fracturing. It typically consists of over 90\% quartz and has a grain size between 0.0625 and 2.0 mm. Further, it must be smooth and round and the distribution must be relatively uniform. The studied geology shows multiple layers of sand containing more than 85\% quartz, with some layers showing percentages of more than 99\%, which helps explain the demand for sand from the study region. Grain size distributions that were created based on taken bed samples do not meet the specifications related to grain uniformity, but it most be noted that only a limited number of river samples were taken and no dry samples were analyzed. Other factors that contribute to the demand for sand are the deltaic nature of the study area, natural river and wind processes help purify and round the grains, and local government policy. As became clear from stakeholder interviews and the literature analysis, sand extraction taxes have been constant for years in spite of strong inflation and are not enough to compensate for road damage that exists due to the mining activities. This makes mining in the area economically as well as technically attractive.

\section{Effects of sand extraction}

Wet:
koppel waargenomen erosie en sedimentbalans aan elkaar -> conclusie, erosie is zoveel maar niet door baggeren
When it comes to riverbank stability, landowners described severe erosion of up to thirty metres per year, which they attributed to the activities of dredging vessels and passing cargo ships. The caretaker added that vegetation removal near the club increased local erosion. In contrast to this, the mayor of Ibicuy downplayed the role of dredging, attributing bank collapses in his jurisdiction to the natural flow of the river.
perception gap

Baggerhoeveelheden zijn gering vergeleken met invoer sediment, dus effecten op oeverstabiliteit lijken ook klein (aanhalen studie Mekong)

Socioeconomic: geluid, daardoor gestopt


When it comes to dry sand mining, volumes are considerably larger and therefore the effects are also more evident and diverse. First of all, dry sand is transported to Añelo, at around 1300 km from Ibicuy, and for this transport, trucks are used. The trucks damage the roads, as became clear from interviews with stakeholders. Many stakeholders complained about the state of the roads, a view that is confirmed by literature and fieldwork observations \autocite{fogliaSedArena2023} \autocite{novasImpactoAmbientalOculto2022}. Although no research on the exact causes of bad road conditions were done in this report, the extent of truck traffic (350 trucks per day from Ibicuy loaded up to around 25 tons each, as per the mayor of Ibicuy) makes this scenario more than likely. During the fieldwork, many cases of road damage, such as potholes, likely caused by heavy vehicles were observed. The economic reality in the region is linked to the road damage. As mentioned, the taxes on sand mining activities were kept constant in the years before 2024, so that because of inflation, the provincial income related to the activities was only 440.000 Euros in 2023. This is enough for repaving around 1 kilometer of road that is in `fair' condition, by U.S. standards \autocite{crumbCostRoadMaintenance2024}. The rate was increased sixfold in 2025, but according to the president of Entre Ríos’ tax authority, this still falls short of covering the province’s road repair costs \autocite{bellatoEntreRiosFrigerio2025}. This underscores that political and economic choices have allowed for current poor road conditions to exist.

Another obvious effect of dry sand mining is related to natural habitats. Several meters of landscape are removed, including all organisms on it. This has the potential to change habitats for good and have a negative impact on biodiversity; U.S. examples of species that were endangered due to sand mining practices already exist \autocite{centerforbiologicaldiversityLegalInterventionLaunched2025}. While visiting the sand mine, it seemed that the natural cover (grasslands and forests) was able to recover within merely a few years in the study area. It must however be noted that changes to the original natural environment can still be severe: the original soil profile has drastically changed, meaning that even though a new natural cover has emerged, original species might not be able to survive anymore. The literature study has yielded further health-related effects: under high concentrations, silica dust can contribute to the development of cancer, which is a risk for workers at sand mines. Previous research from the US has shown that concentrations near mines were too low to pose a risk to citizens, but bans on sand mining in that country have been implemented because of health concerns and were upheld by the supreme court \autocite{petersCommunityAirborneParticulate2017} \autocite{physiciansforsocialresponsibilityCompendiumScientificMedical2023}. The exact effects on the natural habitats and health of citizens in the Lower Paraná Delta thus remain unknown. 

Finally, washing operations by the sand mines require vast amounts of groundwater: in 2020, 400 million liters of groundwater per month from the Talavera formation was used. In comparison, the drinking water cooperation extracts 30 million liters per month in winter and 60 million liters per month in the summer from this reservoir. Considering the further increase in mining activities since 2020, it seems likely that even more groundwater is extracted now. This may lead to drinking water scarcity in the future. The washing operations furthermore pollute the drinking water, which becomes clear from the fact that manganese and iron concentrations have increased significantly, and manganese concentrations are above limits now.

\subsection{3: characteristics of river flow patterns and hydrodynamics}
tides, waves and currents: 

ships: wave load from ships xx
current load

hydrodynamic data: 


\begin{comment}

\section{Interpretation}
Per theme:
Inperpret results
- Identify correlations, patterns, and relationships in data
- Did results meet expectations or hypotheses
- Contextualize findings within previous research and theory
- Explain unexpected results and evaluate their significance
- Consider possible alternative explanations and make an argument for your position
Implication
- Do results support or challenge existing theories and literature? 
- If they support existing theories, what new information do they contribute?
- If they challenge existing theories, why do you think that is?
- Are there any practical implications?

Stakeholders:
Stakeholders identified, with goals and interests
Key players: ports, dredgers, ANPYN
Interview results: two ships active today, river sand for construction, low demand so for fracking likely
Effects: erosion of 30 meters from dredging and cargo
Dry sand mining: 350 trucks with 9000 tons of sand daily, road conditions, supply fracking industry, transported to Anelo
Stakeholders updated: campings more power, dredgers less, ports less interest and power, municipalities introduced. Oppositon increased for landowners and campings

Sand extraction:


Multiple stakeholders, such as the mayor of Ibicuy and the mine manager, explained that mined sand gets transported by trucks to Añelo, a town in Neuquén that forms the heart of the Vaca Muerta fracking activities. This view is confirmed by various reports \autocite{cauceArenasParaFracking2022} \autocite{secretariadepoliticamineraArenasParaFracking2019}.
Geology: borehole, characteristics sand
Consequences: natural habitat, social, economic effects, stakeholders
The most frequently voiced concerns of stakeholders is the poor conditions of roads, especially the provincial RP45 that forms the entrance to the town, caused by intense heavy truck traffic. These comments were confirmed during the field trip, since many potholes were observed along the entire route.

Hydrodynamical and sedimentary analysis:
xx

Delft3D Model:
xx

Mitigation strategies:
xx

4: Limitations
provide an accurate picture of what can and cannot be concluded from your study.
Limitations might be due to research design, methodological choices, or unanticipated obstacles that emerged.
Reiterate why the results are valid for answering research question.

\end{comment}

%verplaatst uit H6

\textcite{schmidtStageDischargeRelationshipOpen2011} note several limitations related to these rating curves. For example, discharge measurements typically scatter and therefore do not show a unique relation with the stage. Also, the underlying physics of the open channel are not captured in the rating. Nevertheless, the relation as given in Equation \ref{eq:powerlaw} is used to approximate dependencies between the variables. 

%Verwijzen naar methode voor fine and course sediments; Ik zou in de discussie willen terugblikken op de methode voor het genereren van time series in het geheel.  En daarbij dit kort herhalen. ;  Overall, the fit gives a good estimate of fine sediment concentrations, revealing seasonal aspects and relevant concentration values.

%6.1.3: It stands out that correlations in the Bermejo are very high for all variables. This indicates that the power-law assumption is suitable. Subsequently, it was found that correlations for the Paraná Guazú are much smaller. This is also the case for the Zárate station on the Paraná de las Palmas. As this river is located in the Lower Paraná delta as well, this behaviour is expected.
% 6.1.3: Analysis of the flow variables has shown that $R^2$-values tend to decrease, the further downstream the location of the measurements is. This result is in agreement with those found by \textcite{songEvaluatingUnderstandingTideriver2024}, who state that the stage-discharge relationship is considerably affected in the transitional zone and tide-dominated region of the Yangtze estuary. Their research reports $R^2$-values decreasing from 0.9 to 0.4, in a river-dominated and transitional zone, respectively. Therefore, an in-depth study of the tide-river interactions is needed. This includes measurements in both the transitional zone and tide-dominated region, to update the stage-discharge relationship. For this study, the influence of tidal effects are therefore handled in Section \ref{sec: tidal forcing}.

% 6.1.3: As for the sediment-discharge relationship, the differences in correlations in Brazo Largo can be explained through the sources of the variables. Whereas the fine sediment concentration predominantly originates from the Bermejo basin, the discharge is dominated by the Paraguay and Paraná. This results in a weak correlation downstream of the confluence of these rivers \autocite{lopezweibelSourcesTemporalDynamics2022}. Another factor that may perturb the correlation of variables in the Lower Paraná, is the inflow of the Uruguay river. The variation in elevations that occur due to the presence of this river, can cause disturbances in the rating curves. 

%6.2.1: 
%The water elevations in Figure \ref{fig:water elevations fieldwork} are expressed with respect to the IGN datum. For that reason, elevations at Puerto Ibicuy are consistently higher than at Brazo Largo on the same time instants. During the two measurement days, water levels drop while discharges increase.
%which appears to contradict the positive correlation between stage and discharge discussed in Section \ref{sec:correlation of variables}. This discrepancy arises because the standard stage-discharge relationship assumes steady flow and acts on longer timescales. In contrast, on short timescales, water levels in the Paraná Guazú are influenced by variations in the Río de la Plata, including backwater effects and transient downstream control. These factors cause a temporary inverse relation between stage and discharge, leading to the observed short-term response. \textcite{jonesExpandedRatingCurve2019} confirm that traditional rating curves fail in the tidal zone, because tidally influenced discharge is more complex than upstream river discharge, causing non-monotonic behaviour. 

%6.2.2: 
%This pattern likely reflects natural sediment sorting processes, where finer particles are carried further by the current, whereas coarser grains settle in deeper or higher-energy parts of the channel.
%It is important to emphasize that the data supporting this observation were collected outside the river section that falls within the area of interest. Consequently, before drawing any conclusions, it is essential to conduct targeted sampling and grain size analysis within the study area. Only by verifying these local conditions, it can be confirmed whether the same coarsening trend applies here as well.
%Sample Rosario: representatief although niet in ons gebied, plus grotere dieptes
