\chapter{Discussion}
%1: Summarize key findings Summarize key results Short, only main question

\section{1: Sand extraction volumes + purpose}
\subsection{Interviews}
River
Interviews: Only two dredgers remain active; dredging has declined sharply due to municipal taxes and administrative restrictions. Dredging operations moved from Ibicuy to other ports (e.g. Port Constanza).

This contradicts the initial hypothesis of an increasing trend in wet sand mining.

The decline in river dredging could temporarily reduce hydromorphological stress (bank erosion, sediment imbalance), but the economic drivers remain present, suggesting that activity could resume if regulations loosen.

Points to the role of local governance (municipalities) as a powerful regulator — a finding often overlooked in similar international cases (e.g., Mekong).

Dry sand mining
Interviews: Massive scale: 120,000 tons/month at YPF, ~350 trucks/day leaving Ibicuy region.
Dominant purpose: fracking sand, not construction.
High-quartz, fine-grained sand from Ibicuy is uniquely suitable for hydraulic fracturing.
Sand mining changing form

AIS data: two ships
Tabel \ref{tab:sand_volume} AIS data: 100000 kuub per maand
Historic AIS data not available, but upward trend in permits not visible, in fact stopped in Ibicuy

Dry sand: data shows upward trend in Vaca Muerta gas, along with this sand mining up
Dry sand: shift from construction to fracking
Likely that river sand can be used for fracking in the future (interview + data)

\subsection{5: Demand for sand}
Expansion of hydraulic fracturing (Vaca Muerta) demanding high-silica, well-sorted sand
Available in entre rios

\subsection{Effects of dredging}
Conflicting views: some see dredging as harmless or natural, others report severe erosion (up to 30 m/year).

Municipal and port officials often downplay dredging impacts compared to residents.
Confirms that perception gaps exist between economic actors (who minimize impacts) and local landowners/fishers (who experience them directly).

\subsection{3: characteristics of river flow patterns and hydrodynamics}
tides, waves and currents: 


\begin{comment}

\section{Interpretation}
Per theme:
Inperpret results
- Identify correlations, patterns, and relationships in data
- Did results meet expectations or hypotheses
- Contextualize findings within previous research and theory
- Explain unexpected results and evaluate their significance
- Consider possible alternative explanations and make an argument for your position
Implication
- Do results support or challenge existing theories and literature? 
- If they support existing theories, what new information do they contribute?
- If they challenge existing theories, why do you think that is?
- Are there any practical implications?

Stakeholders:
Stakeholders identified, with goals and interests
Key players: ports, dredgers, ANPYN
Interview results: two ships active today, river sand for construction, low demand so for fracking likely
Effects: erosion of 30 meters from dredging and cargo
Dry sand mining: 350 trucks with 9000 tons of sand daily, road conditions, supply fracking industry, transported to Anelo
Stakeholders updated: campings more power, dredgers less, ports less interest and power, municipalities introduced. Oppositon increased for landowners and campings

Sand extraction:
River: AIS data shows 2 ships, 3 trips per day
Volumes calculated
Interview results: two ships active today
Dry sand mining: fracking more and more important, upward trend seen as well 
This indicates that the upgoing trend has since persisted.
It seems therefore clear that fracking practices are the driving force behind the increasing demand of sand. This is also the conclusion that other reports reach \autocite{secretariadepoliticamineraArenasParaFracking2019} \autocite{fogliaSedArena2023}.
As mentioned in chapter \ref{chapter:stakeholders}, the mayor of Ibicuy described the following scale during the conducted interview: 350 trucks that transport 9,000 tons of sand each day. With a total of 260 working days per year, this amounts to more than 2.3 million tons of sand transported from Ibicuy in 2025. In 2022: 1.250.000 tons in Ibicuy, so further increase can be seen.
Multiple stakeholders, such as the mayor of Ibicuy and the mine manager, explained that mined sand gets transported by trucks to Añelo, a town in Neuquén that forms the heart of the Vaca Muerta fracking activities. This view is confirmed by various reports \autocite{cauceArenasParaFracking2022} \autocite{secretariadepoliticamineraArenasParaFracking2019}.
Geology: borehole, characteristics sand
Consequences: natural habitat, social, economic effects, stakeholders
The most frequently voiced concerns of stakeholders is the poor conditions of roads, especially the provincial RP45 that forms the entrance to the town, caused by intense heavy truck traffic. These comments were confirmed during the field trip, since many potholes were observed along the entire route.

Hydrodynamical and sedimentary analysis:
xx

Delft3D Model:
xx

Mitigation strategies:
xx

4: Limitations
provide an accurate picture of what can and cannot be concluded from your study.
Limitations might be due to research design, methodological choices, or unanticipated obstacles that emerged.
Reiterate why the results are valid for answering research question.

\end{comment}