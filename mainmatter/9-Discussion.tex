\chapter{Discussion}
%1: Summarize key findings Summarize key results Short, only main question

\section{Sand extraction volumes, purpose and the demand for sand}
Interviews with local stakeholders reveal that dredging activity in recent years has stayed constant in certain parts of the study area. In the Paraná Guazú, two dredging companies are operational, as both AIS data and interview data show. Stakeholders did not speak of a trend in these areas. In Puerto Ibicuy, river sand extraction taxes were increased significantly, leading to the discontinuation of almost all dredging activities on the Paraná Ibicuy. Historic AIS data was not available, but present-day AIS data as well as fieldwork observations confirm that no dredging activities on the Paraná Ibicuy take place anymore. This finding directly contradicts the initial hypothesis suggesting a potential increase in river sand mining. Instead, it highlights the influence of local governance as a powerful regulator of extraction activities.

In contrast, dry sand mining has expanded dramatically. Historic data shows the increasing importance of fracking gas and oil from Vaca Muerta since its discovery in 2011. This trend goes hand-in-hand with the trend in mined sand masses in Argentina, increasing from around 500,000 tons in 2011 to 3.5 million tons in 2020. Although sand is also used by the glass and ceramics industry and the exact demand from different sectors is unknown, the upgoing trend in both data sets is a clear indicator that mined sand is used more and more for fracking. This view is supported by stakeholder interviews and previous research \autocite{fogliaSedArena2023} \autocite{secretariadepoliticamineraArenasParaFracking2019}. As mentioned in Chapter \ref{chapter:stakeholders}, the mayor of Ibicuy described the following scale during the conducted interview: 350 trucks that transport 9,000 tons of sand each day. With a total of 260 working days per year, this amounts to more than 2.3 million tons of sand transported from Ibicuy in 2025. In 2022, the mined mass was 1.250.000 tons in Ibicuy \autocite{fogliaSedArena2023}. This indicates a further growth of the sand mining industry in Ibicuy in recent years, the bulk of which is driven by the fracking practices in the South of the country. This supports initial expectations, although stakeholder interviews and river extraction data indicate that for the area of interest, the increase is due to dry sand mining and not due to river mining. 

However, as becomes clear from the oil and gas production trends in Argentina, fracking forms a greater and greater share of the national production. Fracking in the Neuquén basin is viewed by some as critical to the development of Argentina and the Government of Argentina still views the oil and gas sector as a crucial part of its economy, by driving exports as well as generating foreign currency and investment. It also becomes clear that, considering the recoverable volumes present in the reserves (16 billion barrels of oil and 8722 billion cubic meters of gas), even more resources could still be extracted. The upgoing trend in fracking is expected to increase further and it is estimated that the number of fractures in 2025 will increase by 25\% as compared to 2024 \autocite{barnedaFrackingVacaMuerta2025}. A further increase in fracking will likely lead to more sand extraction from the study area and can ultimately lead to increased sand extraction from the river. After all, stakeholders indicated that river sand can be used for fracking and the dredger mentioned that selling to the fracking industry could be likely.

Fracking sand, also known as silica sand, must meet strict physical and chemical standards to be suitable for use in hydraulic fracturing. It typically consists of over 90\% quartz and has a grain size between 0.0625 and 2.0 mm. Further, it must be smooth and round and the distribution must be relatively uniform. The studied geology shows multiple layers of sand containing more than 85\% quartz, with some layers showing percentages of more than 99\%, which helps explain the demand for sand from the study region. Grain size distributions that were created based on taken bed samples do not meet the specifications related to grain uniformity, but it most be noted that only a limited number of river samples were taken and no dry samples were analyzed. Other factors that contribute to the demand for sand are the deltaic nature of the study area, natural river and wind processes help purify and round the grains, and local government policy. As became clear from stakeholder interviews and the literature analysis, sand extraction taxes have been constant for years in spite of strong inflation and are not enough to compensate for road damage that exists due to the mining activities. This makes mining in the area economically as well as technically attractive.

\section{Hydrodynamic and sedimentary analysis}
The hydrodynamic analysis as carried out in chapter \ref{chap 6: effect and damage on the river bank} provided insights on water elevations, discharge and sediment concentrations for different locations in and near the study area. With the help of correlation matrices, the relationship between these variables was determined. Pre-defined functions were used to define relations between water elevations and discharge. \textcite{schmidtStageDischargeRelationshipOpen2011} note several limitations related to these rating curves, such as the fact that discharge measurements typically scatter and therefore do not show a unique relation with the stage and the ignorance of underlying physics of the open channel in the rating. However, correlations in the Bermejo are high for all variables, indicating that the used power-law assumption is suitable. Subsequently, it was found that correlations for the Paraná Guazú are much smaller. This is also the case for the Zárate station on the Paraná de las Palmas. As this river is located in the Lower Paraná delta as well, this behaviour is expected. Analysis of the flow variables has shown that $R^2$-values tend to decrease, the further downstream the location of the measurements is. This result is in agreement with those found by \textcite{songEvaluatingUnderstandingTideriver2024}, who state that the stage-discharge relationship is considerably affected in the transitional zone and tide-dominated region of the Yangtze estuary. Their research reports $R^2$-values decreasing from 0.9 to 0.4, in a river-dominated and transitional zone, respectively. This pattern is in accordance with the correlations found in the Lower Paraná Delta. Another factor that may perturb the correlation of variables in the Lower Paraná, is the inflow of the Uruguay river. The variation in elevations that occur due to the presence of this river, can cause disturbances in the rating curves. To find out more about factors affecting these relationships, a study of the tide-river interactions was carried out. It follows that the tidal variance fraction is only 5.71\%, which indicates that tidal forcing plays a subordinate role relative to other drivers, such as river discharge. 

Further, for the sediment-discharge relationship, the differences in correlations in Brazo Largo can be explained through the sources of the variables: whereas the fine sediment concentration predominantly originates from the Bermejo basin, the discharge is dominated by the Paraguay and Paraná. This results in a weak correlation downstream of the confluence of these rivers \autocite{lopezweibelSourcesTemporalDynamics2022}. 

In addition to available data, data was gathered during the fieldwork. This consisted flow velocity, discharge, grain size distributions and concentrations of suspended and bed sediments. The measured water elevations, retrieved from public datasets, at Puerto Ibicuy are consistently higher than at Brazo Largo on the same time instants. Further, during the two measurement days, water levels drop while discharges increase. This appears to contradict the positive correlation between stage and discharge discussed in section \ref{sec:correlation of variables}. This discrepancy likely arises because the standard stage-discharge relationship assumes steady flow and acts on longer timescales. In contrast, on short timescales, water levels in the Paraná Guazú are influenced by variations in the Río de la Plata, including backwater effects and transient downstream control. These factors can cause a temporary inverse relation between stage and discharge, leading to the observed short-term response. \textcite{jonesExpandedRatingCurve2019} confirm that traditional rating curves fail in the tidal zone, because tidally influenced discharge is more complex than upstream river discharge, causing non-monotonic behaviour. 

Grain size distribution results reveal that the average particle size of the sand tends to increase with recovery depth. This pattern likely reflects natural sediment sorting processes, where finer particles are carried further by the current, whereas coarser grains settle in deeper or higher-energy parts of the channel. It is important to emphasize that the data supporting this observation were collected outside the area of interest but from near Rosario. However, the Rosario samples are considered representative since they were taken near the study area at a distance of approximately 150 kilometers in the same river. The fact that the Rosario sample shows coarser particles is likely due to the fact that this was taken from a considerably deeper part, at around 30 m depth. 

For the suspended sediment concentrations, it was found that this is the highest near the bed, which is typical for suspended sediment. The steeper profile for cross section 2 indicates higher turbulence or finer particles compared to cross sections 1 and 3. Higher turbulence combined with finer particles causes sediment to stay suspended more easily, which results in a steeper concentration gradient with depth. Combining concentration data with ADCP results yielded the total suspended sediment flux. Similar mean fluxes were found for cross sections 1 and 2, whereas cross Section 3 showed a slightly higher mean velocity. This increased flow velocity can be explained by flow concentration as a result of the nearby confluence. The discharges found were quite small compared to the existing relationship between fine sediment loads and fluvial discharge in Brazo Largo, while the loads of section 1 and section 3 were close to the fit. The fine sediment load measured in section 2 was relatively low. This may underestimate the actual inflow into the confluence, potentially affecting the sediment balance, for which mean suspended sediment fluxes served as the basis.

It must be noted that the calculation of the suspended sediment flux is subject to several sources of uncertainty. Firstly, for each cross section, suspended sediment concentration was measured at only one location. Because no information is available on the grain size distribution, the settling velocity ($w_s$) for the entire cross section was determined by calibrating the Rouse profile to fit this single measurement. Consequently, instead of calculating a separate Rouse profile for each grain size bin, the calibrated $w_s$ was assumed to be representative for all sediment grains across the cross section. The lack of grain size information and the assumption of a representative $w_s$ introduce considerable uncertainty into the results. Secondly, after plotting the depth averaged velocities along the three cross sections, it was found that the velocities show a spread of $\pm$0.2 m/s. The average flow velocity for the three cross sections is around 0.5-0.6 m/s. As the suspended sediment flux is calculated by multiplying the concentration by the velocity, this uncertainty in velocity (approximately  30-40\% of the mean flow) can lead to a comparable relative uncertainty in the calculated suspended sediment flux. In addition, part of the uncertainty may come from short-term flow fluctuations during the measurements and from spatial variability within the cross section that is not fully captured by the ADCP transects. The computed sediment deficit is mostly caused by the ingoing and outgoing suspended sediment fluxes, so over or underestimation of these fluxes can suggest an artificial sediment deficit. This needs to be taken into account when interpreting the sediment balance and its conclusions, but in spite of its uncertainties, the conclusions derived from the sediment balance are deemed valid. After all, any resulting effects do not depend on exact sediment values and in fact an estimation with a correct order of magnitude is likely enough to derive conclusions. 

\section{The effects of sand extraction}
The effects of sand extraction from the delta area are a key concern of this research. From chapter \ref{chapter:background}, it is clear that river sand mining can for example cause bank instability and erosion. In stakeholder interviews, erosion scales of 30 meters per year were named. Through analysis of satellite data, values ranging between 57 m and 138 m were found for a period of 20 years. This gives considerably lower scales of erosion, between around 3 and 7 meters per year, but is nonetheless significant. In stakeholder interviews, erosion was explicitly linked to dredging activities. The sediment balance can provide insights on the likelihood of these statements. In this calculation, a net negative sediment flux of 19,172.88 tonnes/day was found. Compared to the estimated amount of extracted sand, equal to 5333.33 tonnes/day, it appears unlikely that the net erosion is primarily caused by sand extraction. Even without sand extraction, the system would still lack sufficient sediment to balance the deficit.

As discussed in chapter \ref{chapter:background}, a possible negative effect of river sand extraction is disruption of the river bed. Mining pits are created and later extend in downstream as well as upstream directions. This effect was not found during field work measurement, since longitudinal profiles revealed no significant sand dunes or bed forms. This discrepancy with existing literature can be attributed to the scale of operations: from the sediment balance it follows that dredged volumes (5333.33 tonnes per day) are about one fourth of the sediment influx (20986.56 tonnes per day). In the lower Mekong river on the other hand, extraction rates are about 8 times greater than the total sand flux entering the delta \autocite{hackneyRiverBankInstability2020}. As mentioned in chapter \ref{chapter:background}, the same study showed that dredging can cause river bed levels to be lowered sufficiently to cause river bank instability. The scale of sand mining activities together with the absence of mining pits indicates that dredging-induced river bank instability is not a concern for the Lower Paraná delta at present.

A final, socioeconomic, effect of river sand mining came forward in stakeholder interviews. Dredging vessels cause noise nuisance, which forms a disturbance for people near the river shores. Because of complaints by campings that are of economic importance to the region, dredging activities were in fact stopped in Ibicuy.

When it comes to dry sand mining, volumes are considerably larger and therefore the effects are also more evident and diverse. First of all there is the truck damage, as became clear from interviews with stakeholders. Many stakeholders complained about the state of the roads, a view that is confirmed by literature and fieldwork observations \autocite{fogliaSedArena2023} \autocite{novasImpactoAmbientalOculto2022}. Although no research on the exact causes of bad road conditions were done in this report, the extent of truck traffic (350 trucks per day from Ibicuy loaded up to around 25 tons each, as per the mayor of Ibicuy) makes this scenario more than likely. The economic reality in the region is linked to the road damage. As mentioned, the taxes on sand mining activities were kept constant in the years before 2024, so that because of inflation, the provincial income related to the activities was only 440.000 Euros in 2023. This is enough for repaving around 1 kilometer of road that is in `fair' condition, by U.S. standards \autocite{crumbCostRoadMaintenance2024}. The rate was increased sixfold in 2025, but according to the president of Entre Ríos’ tax authority, this still falls short of covering the province’s road repair costs \autocite{bellatoEntreRiosFrigerio2025}. This underscores that political and economic choices have allowed for current poor road conditions to exist.

Another obvious effect of dry sand mining is related to natural habitats. Several meters of landscape are removed, including all organisms on it. This has the potential to change habitats for good and have a negative impact on biodiversity; U.S. examples of species that were endangered due to sand mining practices already exist \autocite{centerforbiologicaldiversityLegalInterventionLaunched2025}. While visiting the sand mine, it seemed that the natural cover (grasslands and forests) was able to recover within merely a few years in the study area. It must however be noted that changes to the original natural environment can still be severe: the original soil profile has drastically changed, meaning that even though a new natural cover has emerged, original species might not be able to survive anymore. The literature study has yielded further health-related effects: under high concentrations, silica dust can contribute to the development of cancer, which is a risk for workers at sand mines. Previous research from the US has shown that concentrations near mines were too low to pose a risk to citizens, but bans on sand mining in that country have been implemented because of health concerns and were upheld by the supreme court \autocite{petersCommunityAirborneParticulate2017} \autocite{physiciansforsocialresponsibilityCompendiumScientificMedical2023}. Exact effects on the natural habitats and health of citizens in the Lower Paraná Delta therefore do not become clear from this report. 

Finally, washing operations by the sand mines require vast amounts of groundwater: in 2020, 400 million liters of groundwater per month from the Talavera formation was used \autocite{cauceArenasParaFracking2022}. In comparison, the drinking water cooperation extracts 30 million liters per month in winter and 60 million liters per month in the summer from this reservoir \autocite{fogliaSedArena2023}. Considering the further increase in mining activities since 2020, it seems likely that even more groundwater is extracted now. This may lead to drinking water scarcity in the future. The washing operations furthermore pollute the drinking water, which becomes clear from the fact that manganese and iron concentrations have increased significantly, and manganese concentrations are above limits now.

Based on derived effects of river and dry sand mining, a number of mitigation strategies was proposed. X and X were most useful and they help prevent effects Y and Y as became evident from the analysis.

As became clear from 



%Mitigations

%Limitatie in het algemeen: toekomstscenarios
\section{Limitations}
For the presented study, a number of limitations must be acknowledged.

Andere limitations: alleen huidige situ

Veldwerk: 2 dagen metingen

%inkorten: dingen naar conclusie?






%6.1.2 sediment loads: Verwijzen naar methode voor fine and coarse sediments; Ik zou in de discussie willen terugblikken op de methode voor het genereren van time series in het geheel.  En daarbij dit kort herhalen. ;  Overall, the fit gives a good estimate of fine sediment concentrations, revealing seasonal aspects and relevant concentration values.


%6.2.4:
%\subsubsection{Uncertainties}
%The uncertainties in estimating bed load transport are primarily related to the composition of the bed material. Engelund-Hansen is used here, which is predominantly suited for sandy beds. As observed during field measurements, the Paraná Guazú contains significant amounts of fine sediments, such as clays and silts. In some samples, no sand was detected; however, this largely depended on the sampling depth—at deeper locations, sand was indeed present. Therefore, the application of Engelund-Hansen is justified, and it can be assumed that section 3 also contains sand. The sediment diameter is assumed to be similar to that reported by \textcite{reMetodologiaParaGeneracion2009}.

\subsection{Qualitative Results}
From this data, one can see that for the camping there has been a loss of 57 to 138 meters on the bank of the land. Some parts are more extreme than others, but on the whole from 2003 to 2022 this amount of land loss is significant.

First of all, it is important to ask how this happened and what factor or problem contributes the most to the calculated erosion. The group started from a quite biased point of view only taking into account the opinion of the stakeholders. They gave away that the reason for this erosion was purely the dredging of the sand and the cargo ships passing by. From the Wave impact study it shows that the cargo ships induced waves do in fact contribute to the bank erosion when passing by. It is quite hard to determine how much faster the erosion is happening due to this event but one might assume the fact that the contribution to the problem is small, but not insignificant. Moreover, it is important to take into account the fact that the likelihood of the digging for a boat channel for the camping adds onto the erosion of the shore.

The Aqua Monitor study illustrates the high correlation between the water levels and discharge intensities with the water quantities gained in certain areas. The borders of curves in a channel are the most likely to be victim of the water gains which makes the bank stability decrease when the hydro parameters discharge and water quantity are restored to a normal level, as explained in the section \ref{chap 6: effect on the river bed}. From this part, one can assume that after a flood, the banks become even more fragile and break apart at a faster pace after the flood is gone. In addition to that, one can also state that this major factor contributing to bank erosion as the satellite data shows in Figure \ref{fig:surface_comparison}.

It is also relevant to note that the number of flooding occurrences is positively correlated with climate change. This is because warmer temperatures cause more water to evaporate from the land and oceans, as well as changes in the size and frequency of heavy precipitation events. Thus, more extreme droughts, evaporation, and precipitations lead to more extreme floods \autocite{usenvironmentalprotectionagencyClimateChangeIndicators2016}.

Lastly, the natural erosion of a river has to be taken seriously when analysing this problem. Since the study area part of the river has meanders, it is normal to have some erosion on the outsides of the curves and deposition on the inside of the turns, due to the water with sediment passing by. The reason behind it is that water flows fast on the outside, leading to more erosion. On the inside, the water flows slowly, and thus making it easier for sediment to be deposited. Overtime, this creates more wider turns, explaining the erosion on the outside curve near the Camping of La Blanqueada \autocite{serenaniWhyRiversCurve2025}.

Therefore, climate change and natural erosion are the significant factors that contribute to bank erosion, and not the passing of cargo ships, nor the dredging of the sand a few hundred meters from the Camping La Blanqueada.
The situation was analysed for this very precise location. Other parameters may be taken into account when establishing this reasoning on another section of the Parana Guazu, but the method can be the same and the satellite images as well as the Aqua Monitor are accessible for every location. 