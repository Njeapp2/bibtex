\chapter{Discussion}
%1: Summarize key findings Summarize key results Short, only main question

\section{Sand extraction volumes, purpose and the demand for sand}
Interviews with local stakeholders reveal that dredging activity in recent years has stayed concstant in certain parts of the study area. In the Paraná Guazú, two dredging companies are operational, as both AIS data and interview data show. Stakeholders did not speak of a trend in these areas. In Puerto Ibicuy, river sand extraction taxes were increased significantly, leading to the discontinuation of almost all dredging activities on the Paraná Ibicuy. Historic AIS data was not available, but present-day AIS data as well as fieldwork observations confirm that no dredging acitivies on the Paraná Ibicuy take place anymore. This finding directly contradicts the initial hypothesis suggesting a potential increase in river sand mining. Instead, it highlights the influence of local governance as a powerful regulator of extraction activities.

In contrast, dry sand mining has expanded dramatically. Historic data shows the increasing importance of fracking gas and oil from Vaca Muerta since its discovery in 2011. This trend goes hand-in-hand with the trend in mined sand masses in Argentina, increasing from around 500,000 tons in 2011 to 3.5 million tons in 2020. Although and is also used by the glass and ceramics and the exact demand of sand from different sectors is unknown, the upgoing trend in both data sets is a clear indicator that mined sand is used more and more for fracking. This view is supported by stakeholder interviews and previous research \autocite{fogliaSedArena2023} \autocite{secretariadepoliticamineraArenasParaFracking2019}. This supports initial expectations that increased mining volumes can be attributed to fracking, although stakeholder interviews and river extraction data indicate that for the area of interest, this increase is due to dry sand mining and not due to river mining.

However, as becomes clear from the oil and gas production trends in Argentina, fracking forms a greater and greater share of the national production. Fracking in the Neuquén basin is viewed by some as critical to the development of Argentina and the Government of Argentina still views the oil and gas sector as a crucial part of its economy, by driving exports as well as generating foreign currency and investment. It also becomes clear that, considering the recoverable volumes present in the reserves (16 billion barrels of oil and 8722 billion cubic meters of gas), even more resources could still be extracted. The upgoing trend in fracking is expected to increase further and it is estimated that the number of fractures in 2025 will increase by 25\% as compared to 2024 \autocite{barnedaFrackingVacaMuerta2025}. A further increase in fracking will likely lead to more sand extraction from the study area and can ultimately lead to increased sand extraction from the river. After all, stakeholders indicated that river sand can be used for fracking and the dredger mentioned that selling to the fracking industry could be likely.

\subsection{5: Demand for sand}
Expansion of hydraulic fracturing (Vaca Muerta) demanding high-silica, well-sorted sand
Available in entre rios

\subsection{Effects of dredging}
Conflicting views: some see dredging as harmless or natural, others report severe erosion (up to 30 m/year).

Municipal and port officials often downplay dredging impacts compared to residents.
Confirms that perception gaps exist between economic actors (who minimize impacts) and local landowners/fishers (who experience them directly).

\subsection{3: characteristics of river flow patterns and hydrodynamics}
tides, waves and currents: 

ships: wave load from ships xx
current load

hydrodynamic data: 


\begin{comment}

\section{Interpretation}
Per theme:
Inperpret results
- Identify correlations, patterns, and relationships in data
- Did results meet expectations or hypotheses
- Contextualize findings within previous research and theory
- Explain unexpected results and evaluate their significance
- Consider possible alternative explanations and make an argument for your position
Implication
- Do results support or challenge existing theories and literature? 
- If they support existing theories, what new information do they contribute?
- If they challenge existing theories, why do you think that is?
- Are there any practical implications?

Stakeholders:
Stakeholders identified, with goals and interests
Key players: ports, dredgers, ANPYN
Interview results: two ships active today, river sand for construction, low demand so for fracking likely
Effects: erosion of 30 meters from dredging and cargo
Dry sand mining: 350 trucks with 9000 tons of sand daily, road conditions, supply fracking industry, transported to Anelo
Stakeholders updated: campings more power, dredgers less, ports less interest and power, municipalities introduced. Oppositon increased for landowners and campings

Sand extraction:
River: AIS data shows 2 ships, 3 trips per day
Volumes calculated
Interview results: two ships active today
Dry sand mining: fracking more and more important, upward trend seen as well 
This indicates that the upgoing trend has since persisted.
It seems therefore clear that fracking practices are the driving force behind the increasing demand of sand. This is also the conclusion that other reports reach \autocite{secretariadepoliticamineraArenasParaFracking2019} \autocite{fogliaSedArena2023}.
As mentioned in chapter \ref{chapter:stakeholders}, the mayor of Ibicuy described the following scale during the conducted interview: 350 trucks that transport 9,000 tons of sand each day. With a total of 260 working days per year, this amounts to more than 2.3 million tons of sand transported from Ibicuy in 2025. In 2022: 1.250.000 tons in Ibicuy, so further increase can be seen.
Multiple stakeholders, such as the mayor of Ibicuy and the mine manager, explained that mined sand gets transported by trucks to Añelo, a town in Neuquén that forms the heart of the Vaca Muerta fracking activities. This view is confirmed by various reports \autocite{cauceArenasParaFracking2022} \autocite{secretariadepoliticamineraArenasParaFracking2019}.
Geology: borehole, characteristics sand
Consequences: natural habitat, social, economic effects, stakeholders
The most frequently voiced concerns of stakeholders is the poor conditions of roads, especially the provincial RP45 that forms the entrance to the town, caused by intense heavy truck traffic. These comments were confirmed during the field trip, since many potholes were observed along the entire route.

Hydrodynamical and sedimentary analysis:
xx

Delft3D Model:
xx

Mitigation strategies:
xx

4: Limitations
provide an accurate picture of what can and cannot be concluded from your study.
Limitations might be due to research design, methodological choices, or unanticipated obstacles that emerged.
Reiterate why the results are valid for answering research question.

\end{comment}