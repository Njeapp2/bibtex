\chapter{Mitigation strategies}


\section{Nature-based Solutions}
When it comes to Nature-based solutions, the question arises what this definition means. A quite general definition of nature based solutions would be;

\textit{“Nature-based Solutions are actions to protect, conserve, restore, sustainably
use and manage natural or modified terrestrial, freshwater, coastal and marine
ecosystems, which address social, economic and environmental challenges
effectively and adaptively, while simultaneously providing human well-being,
ecosystem services, resilience and biodiversity benefits” (United Nations, 2022,
p. 2)}

Although this definition may give a thought that it only concerns natural and biodiversity increasing ideas this is actually not the case. For example the impact of NBS on the local economies and communities is of an equal importance. When it comes to weighing the different NBS against each other this report will make use of the seven goals of the IUCN which must be achieved as good as possible. The seven goals are presented below;

\begin{figure}[H]
    \centering
    \includegraphics[width=0.50\linewidth]{figures/ThesevenNBSgoals.png}
    \caption{Seven goals for achieving a good NBS}
    \label{fig:placeholder}
\end{figure}

\subsection{Resistance against NBS}
Although NBS are widely known in the scientific world, most people have never heard of these solutions. So, when implementing a solution which can't be described as a classical solution, there is a big change of getting resistance from multiple stakeholders. Especially local communities are skeptical because the solution is less concrete than 'normal' solution would be. The business case of a NBS must of course also be solid. Without funding of the project, there will never be a change to realize it. That's why it's from great importance to have a solution which is both profitable as well explained to the stakeholders. 

\subsection{Implementing NBS in this project}


\section{Bed and bank protection measures}
