\chapter{Mitigation strategies}


\section{Nature-based Solutions}
When it comes to Nature-based solutions, the question arises what this definition means. A quite general definition of nature based solutions would be;

\textit{“Nature-based Solutions are actions to protect, conserve, restore, sustainably
use and manage natural or modified terrestrial, freshwater, coastal and marine
ecosystems, which address social, economic and environmental challenges
effectively and adaptively, while simultaneously providing human well-being,
ecosystem services, resilience and biodiversity benefits” (United Nations, 2022,
p. 2)}

Although this definition may give a thought that it only concerns natural and biodiversity increasing ideas this is actually not the case. For example the impact of NBS on the local economies and communities is of an equal importance. When it comes to weighing the different NBS against each other this report will make use of the seven goals of the IUCN which must be achieved as good as possible. The seven goals are presented below;

\begin{figure}[H]
    \centering
    \includegraphics[width=0.50\linewidth]{figures/ThesevenNBSgoals.png}
    \caption{Seven goals for achieving a good NBS}
    \label{fig:placeholder}
\end{figure}

\subsection{Resistance against NBS}
Although NBS are widely known in the scientific world, most people have never heard of these solutions. So, when implementing a solution which can't be described as a classical solution, there is a big change of getting resistance from multiple stakeholders. Especially local communities are skeptical because the solution is less concrete than a classical solution would be. The business case of a NBS must of course also be solid. Without funding of the project, there will never be a change to realize it. That's why it's from great importance to have a solution which is both profitable as explainable to the stakeholders. 

\subsection{Implementing NBS in this project}
As stressed out before in this report, is there a problem with riverbank erosion due to activities on the river. To mitigate or even solve this problem, it is our intend to use a nature based solution. Therefore there are presented multiple possible solutions to create a long-lasting, sustainable riverbank which is made cost effective. 

One of these solution 





\section{Bed and bank protection measures}

belangrijke/interessante info over deltas en wetlands:
The Paraná Delta, the end of the Paraná-Paraguay river wetland system, begins in the city of Diamante in Entre Ríos province. It stretches for 300 kilometres and covers some 2.3 million hectares. Dotted with islands, these wetlands are a source of ecosystem services such as flood and drought buffering, water purification, erosion control and coastal protection, climate regulation, as well as the provision of shelter, feeding and breeding sites for various wildlife species. It also provides resources including fish, foraging, timber, medicine, and materials for construction and clothing.

In recent years, wetlands have become increasingly important for another key reason: their role as allies against climate change. They improve the resilience of communities to its impacts, serve as natural barriers against floods and droughts, and also function as carbon sinks. Despite playing these important roles, these ecosystems remain under great threat from human action – it is estimated that globally, 85 percent of the wetlands that existed three centuries ago have been destroyed or drastically transformed. 

https://dialogue.earth/en/climate/on-the-parana-river-ecological-crisis-is-a-threat-to-its-identity/


