\chapter{Delft3D Model}
\label{chap:Delft3DModel}
This chapter describes the simulations that were made using Delft3D software from Deltares. The input parameters are discussed, as well as the process of setting up the model and the output related to previously found behaviour of the Paraná Guazú. 

\section{Two-dimensional hydrodynamic model}
First, the hydrodynamic conditions of the study area are simulated by setting up a 2D model. For this purpose, the Delft3D 4 Suite version is applied. This software uses structured grids in which the geometry of the cells is fixed. The two-dimensional approach implies that for every cell, depth-averaged flow conditions govern. It is possible to expand to a 3D model by defining sub-layers over the depth profile. For the sake of simplicity, and the level of detail needed in this study, the 2D model is considered sufficient here. 

\subsection{Grid and bathymetry definition}
The grid is defined similar to Figure \ref{fig:cross section domain}, including an extension of the Talabera branch. This ensures reliable computations in the confluence of the Paraná Guazú and Talabera. The bathymetry is provided by INA through a  


\subsection{Input for hydrodynamic simulation (Delft3D-FLOW)}
