\chapter{Delft3D Model}
\label{chap:Delft3DModel}
This chapter describes the simulations that were made using Delft3D software from Deltares. The input parameters are discussed, as well as the process of setting up the model and the output related to previously found behaviour of the Paraná Guazú. 

\section{Two-dimensional hydrodynamic model}
First, the hydrodynamic conditions of the study area are simulated by setting up a 2D model. For this purpose, the Delft3D 4 Suite version is applied. This software uses structured grids in which the geometry of the cells is fixed. The two-dimensional approach implies that for every cell, depth-averaged flow conditions govern. It is possible to expand to a 3D model by defining sub-layers over the depth profile. For the sake of simplicity, and the level of detail needed in this study, the 2D model is considered sufficient here. 

\subsection{Grid and bathymetry definition}
The grid is defined similar to Figure \ref{fig:cross section domain}, including an extension of the Talabera branch. This ensures reliable computations in the confluence of the Paraná Guazú and Talabera. The \textit{RFRGRID}-tool was applied to generate the grid, after importing land boundaries and manually drawing splines. The grid was then refined twice, using a refinement factor of 3. 

The bathymetry is provided by INA through a Digital Elevation Model from 2019. This was converted to a file containing geospatial information and corresponding depths. The DEM has been visually compared to the ADCP cross sections that result from the field campaign, and it was found that the DEM is representative for the current bathymetry of the river. By making use of the \textit{QUICKIN}-tool, the data is interpolated to the computational grid. This result was found by applying the following operations:

\begin{itemize}
    \item Grid Cell Averaging with 1 as a minimum number of averaging points. This method is preferred over Triangular Interpolation, which is computationally more expensive. In addition, the Grid Cell Averaging approach is suitable when there are more samples than grid points (\textbf{QUICKIN reference}). This is found to be the case when mapping the DEM raster data to a grid. 
    \item Internal Diffusion with 1000 internal diffusion steps
\end{itemize}


\subsection{Input for hydrodynamic simulation (Delft3D-FLOW)}
