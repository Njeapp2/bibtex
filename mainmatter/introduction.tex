\chapter{Introduction}
\label{chapter:introduction}

Chapter 1.1 presents the motivation for the project, outlining the reasons behind its initiation. Chapter 1.2 then provides a detailed description of the project, followed by an outline of its primary objectives, specifying what the project seeks to accomplish. Finally, Chapter 1.3 discusses the methodology that will be applied to achieve these objectives.

\subsection{Project motivation}
Sand is an important resource which, as it is used in countless materials, has shaped our modern world. Therefore, sand is mined throughout the world. One of these places is the Parana delta in Argentina on the border with Uruguay. Although the sand is needed for development of the area, the question also arises what the consequences of the sand mining is for the river, her biodiversity and the surrounding areas. 

This project aims to find out what the magnitude is of the sand mining and what the consequences are. With this information, it is the aim to improve the conditions in which the sand is mined, for the local communities as the biodiversity, and the contractors itself.

\subsection{Problem analysis}




\subsection{Methodology}























It is now well understood that continued and indiscriminate sand mining can cause irreparable and irreversible damages to the ecological and socioeconomic environments of the region,

INTRO aan het einde schrijven.

Sand is a precious resource used in abundance in our society: construction, glass, phones, etc etc bla bla bla. (explain why there is a need).


"Despite the fact that sand is renewable in the geologic time periods, it is considered a nonrenewable resource as its regeneration is meager in the human calendar years. As the sand and gravel resources are extracted easily from the in channel or near-channel sources, people depend on the river sources of sand greatly compared to the other aggregate sources."
(from the sand mining impacts book)
can be used to explain why we take sand from the rivers.

