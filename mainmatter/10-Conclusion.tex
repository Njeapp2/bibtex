\chapter{Conclusion and recommendations}
\label{chapter:conclusion}

This conclusion brings together the findings from the different sub-questions to provide a comprehensive answer to the main research question. It reflects on the outcomes of the study and gives recommendations how further research topics can be explored from this report on.





%Recommendations: health concerns (acrylamide and silica dust) must be sesearched, effects of dry sand mining on nature

%Fracking sand, also known as silica sand, must meet strict physical and chemical standards to be suitable for use in hydraulic fracturing: it typically consists of over 90\% quartz and has a grain size between 0.0625 and 2.0 mm. Further, it must be smooth and round and the distribution must be relatively uniform.


This helps to at least partially explain the erosion on the outside curve near the Camping of La Blanqueada. From this it seems likely that natural erosion and floods, which occur more frequently due to climate change, are the main factors driving erosion.

The scale of sand mining activities together with the absence of mining pits indicates that dredging-induced river bank instability is not a concern for the Lower Paraná delta at present.

 A final, socioeconomic, effect of river sand mining came forward in stakeholder interviews. Dredging vessels cause noise nuisance, which forms a disturbance for people near the river shores. Because of complaints by campings that are of economic importance to the region, dredging activities were in fact stopped in Ibicuy.


This underscores that political and economic choices have allowed for current poor road conditions to exist.

Exact effects on the natural habitats and health of citizens in the Lower Paraná Delta therefore do not become clear from this report.

Based on derived effects of river and dry sand mining, a number of mitigation strategies was proposed. X and X were most useful and they help prevent effects Y and Y as became evident from the analysis.