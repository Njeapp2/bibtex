\chapter{Conclusion and recommendations}
\label{chapter:conclusion}
This chapter brings together the findings from the different sub-questions to provide a comprehensive answer to the main research question. It reflects on the outcomes of the study and gives recommendations how further research topics can be explored from this report on.

\section{Conclusion}
The sub questions help to formulate at an answer to the main research question. The first sub question was related to the sand extraction volumes in the Lower Paraná Delta and the purposes of sand. In the Lower Paraná Delta, river sand extraction has remained relatively constant in most areas, while local government intervention has halted dredging in the Paraná Ibicuy. AIS data indicate that current river sand extraction volumes are around 587,520 tons per year,ry sand mining volumes are approximately 2.3 million tons in 2025 according to local authorities.  River sand is mostly used by the construction sector, which shows relatively low demand, whereas the bulk of dry sand is extracted for fracking activities. Hence, there is an observed upward trend in sand mining activities 


Stakeholders show dredging constant
Local government intervention has stopped dredging in ibicuy
AIS data show present monthly dredging volumes at the river are around 30600 ~m\textsuperscript{3}. 
Dry sand mining activities have increased drastically, 1250000 tons in 2022 and now 2.3 million tons, as per mayor interview.
For the area of interest, the increase is due to dry sand mining and not due to river mining.
River sand was used for the construction sector, low demand -> dry sand is primarily for fracking




This indicates a further growth of the sand mining industry in Ibicuy in recent years, the bulk of which is driven by the fracking practices in the South of the country.

Sediment balance established and quantified by xx
Result: net negative

This makes mining in the area economically as well as technically attractive.


Low compared to influx




%Recommendations: health concerns (acrylamide and silica dust) must be sesearched, effects of dry sand mining on nature

%Fracking sand, also known as silica sand, must meet strict physical and chemical standards to be suitable for use in hydraulic fracturing: it typically consists of over 90\% quartz and has a grain size between 0.0625 and 2.0 mm. Further, it must be smooth and round and the distribution must be relatively uniform.


This helps to at least partially explain the erosion on the outside curve near the Camping of La Blanqueada. From this it seems likely that natural erosion and floods, which occur more frequently due to climate change, are the main factors driving erosion.

The scale of sand mining activities together with the absence of mining pits indicates that dredging-induced river bank instability is not a concern for the Lower Paraná delta at present.

 A final, socioeconomic, effect of river sand mining came forward in stakeholder interviews. Dredging vessels cause noise nuisance, which forms a disturbance for people near the river shores. Because of complaints by campings that are of economic importance to the region, dredging activities were in fact stopped in Ibicuy.


This underscores that political and economic choices have allowed for current poor road conditions to exist.

Exact effects on the natural habitats and health of citizens in the Lower Paraná Delta therefore do not become clear from this report.

Based on derived effects of river and dry sand mining, a number of mitigation strategies was proposed. X and X were most useful and they help prevent effects Y and Y as became evident from the analysis.

