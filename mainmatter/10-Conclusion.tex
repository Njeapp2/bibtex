\chapter{Conclusion and recommendations}
\label{chapter:conclusion}
This chapter brings together the findings from the different sub-questions to provide a comprehensive answer to the main research question. It reflects on the outcomes of the study and gives recommendations how further research topics can be explored from this report on.

\section{Conclusion}
The sub questions help to formulate at an answer to the main research question. The first sub question was related to the sand extraction volumes in the Lower Paraná Delta and the purposes of sand. In the Lower Paraná Delta, river sand extraction has remained relatively constant, while local government intervention has halted dredging in the Paraná Ibicuy. AIS data indicate that current river sand extraction volumes are around 587,520 tons per year, whereas dry sand mining has increased drastically to approximately 2.3 million tons in 2025, about twice the amount extracted in 2022. River sand is mainly used by the construction sector, which has relatively low demand, while most dry sand is extracted for fracking. Overall, sand mining activities show an upward trend, driven primarily by dry sand mining for fracking.

The sediment balance of the relevant section of the Paraná Guazú River was established by quantifying sediment inputs, outputs, and storage within the river reach. During fieldwork, water samples were collected at key points along the river, and the bathymetry was recorded. This was combined with flow velocity and water elevation data to calculate sediment fluxes, which enabled an estimation of the net sediment balance. From this, a net negative sediment flux of 15,369.55 tons per day was found.


This makes mining in the area economically as well as technically attractive.


Low compared to influx




%Recommendations: health concerns (acrylamide and silica dust) must be sesearched, effects of dry sand mining on nature

%Fracking sand, also known as silica sand, must meet strict physical and chemical standards to be suitable for use in hydraulic fracturing: it typically consists of over 90\% quartz and has a grain size between 0.0625 and 2.0 mm. Further, it must be smooth and round and the distribution must be relatively uniform.


This helps to at least partially explain the erosion on the outside curve near the Camping of La Blanqueada. From this it seems likely that natural erosion and floods, which occur more frequently due to climate change, are the main factors driving erosion.

The scale of sand mining activities together with the absence of mining pits indicates that dredging-induced river bank instability is not a concern for the Lower Paraná delta at present.

 A final, socioeconomic, effect of river sand mining came forward in stakeholder interviews. Dredging vessels cause noise nuisance, which forms a disturbance for people near the river shores. Because of complaints by campings that are of economic importance to the region, dredging activities were in fact stopped in Ibicuy.


This underscores that political and economic choices have allowed for current poor road conditions to exist.

Exact effects on the natural habitats and health of citizens in the Lower Paraná Delta therefore do not become clear from this report.

Based on derived effects of river and dry sand mining, a number of mitigation strategies was proposed. X and X were most useful and they help prevent effects Y and Y as became evident from the analysis.

