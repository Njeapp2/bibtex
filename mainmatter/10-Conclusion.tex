\chapter{Conclusion and recommendations}
\label{chapter:conclusion}
\section{Conclusion}
The sub-questions help formulate an answer to the main research question. The first sub-question was related to sand extraction volumes in the Lower Paraná Delta and the purposes of sand. In the Lower Paraná Delta, river sand extraction has remained relatively constant, while local government intervention has halted dredging in the Paraná Ibicuy. AIS data indicate that current river sand extraction volumes are around 587,520 tons per year, whereas dry sand mining has increased drastically to approximately 2.3 million tons in 2025, about twice the amount extracted in 2022. River sand is mainly used by the construction sector, which currently shows relatively low demand, while most dry sand is extracted for fracking. Overall, sand mining activities show an upward trend, driven primarily by dry sand mining for fracking purposes.

The sediment balance of the relevant section of the Paraná Guazú River was established by quantifying sediment inputs, outputs, and storage within the river reach. During fieldwork, sediment samples were collected at key points along the river, flow properties were measured and the bathymetry was recorded. This method delivered velocity profiles that allowed estimations for suspended sediment loads, as well as granulometries and discharge data that served as the input for the Engelund-Hansen equation to estimate bed load transport rates. Subsequently, these fluxes enabled an estimation of the net sediment balance. From this, a net negative sediment flux of 15,369.55 tons per day was found.

%What are the characteristics of river flow patterns and hydrodynamics in the study area?
%Results from Chapter 6/7: weak stage-discharge relationship, relative importance of astronomical/meteorological tides, locations of increased flow velocities

Hydrodynamic analysis shows that stage-discharge relationships are weakly correlated in the study area. This is attributed to variations of the Río de la Plata, which are transient effects that act on different timescales than those under the assumption of steady flow. In addition, the tidal variance fraction of 5.71\% suggests that the influence of the astronomical tide is small relative to the meteorological tide. Measurements and simulations record mean flow velocities between 0.4 and 0.8 m/s, of which the upper values are mostly encountered after the confluences in the study area. Finally, meandering effects in the river cause high flow velocities in outer bends that result in bank erosion, which is confirmed by the Delft3D simulation. 
 
Stakeholders suggested high erosion rates in the study area of up to 30 m per year that were attributed to dredging and cargo ships. Satellite data indicate lower rates of around 3–7 m per year. The sand extraction mass was estimated to be 1,530 tons/day and compared to the total net negative sediment flux, it becomes clear that dredging in the area is not substantial enough to cause a significant portion of observed erosion. Ship-induced waves also appear to play a minimal role, as their force is relatively low and vessel frequency is limited. Instead, erosion is strongly influenced by natural factors, including bank instability following floods, which are expected to increase with climate change. The river’s meandering behavior, which promotes erosion on outer bends and deposition on inner bends, is another key factor in explaining erosion.

Another possible effect of river sand mining is induced riverbank instability due to lowering of the bed. The scale of sand mining activities together with the observed absence of mining pits indicate that dredging-induced river bank instability is not a concern for the Lower Paraná Delta at present. A socioeconomic effect of river sand mining came forward in stakeholder interviews: dredging vessels cause noise nuisance, which forms a disturbance for people near the river shores. This caused activities near Ibicuy to stop. Dry sand mining has more evident and diverse impacts than river sand mining due to larger volumes extracted. Truck traffic causes significant road damage while taxes have not been high enough to fund repairs. This underscores that political and economic choices have allowed for current poor road conditions to exist. Washing operations demand vast amounts of groundwater that far exceed local drinking water use and can pollute water with elevated manganese and iron concentrations. Finally, large-scale removal of soil alters natural habitats, but the exact long-term effects on biodiversity in the study area do not become clear from this research. Sand mining has the potential to have adverse health effects, but the effects on people in the delta were also not determined.

As mentioned before, the rising demand for sand is driven by fracking activities in the south of Argentina. The specific demand for sand from the Lower Paraná Delta can be explained by the sand characteristics that are needed for fracking activities. Fracking sand must meet strict physical and chemical standards: it typically consists of over 90\% quartz, grains must be smooth and round, and the distribution must be relatively uniform. The studied geology shows multiple layers of sand containing more than 85\% quartz, with some layers showing contents of more than 99\%. Grain size distributions that were created based on taken bed samples do not meet the fracking specifications related to grain uniformity, but only a limited number of river samples were taken and no dry samples were analyzed. Other factors that contribute to the demand for sand are the deltaic nature of the study area, natural river and wind processes help purify and round the grains, and local government policy, most notably low taxes. This makes mining in the area economically as well as technically attractive.

A number of observed negative effects call for mitigation measures. To make sure designs align with goals of climate resilience and sustainable development, the focus was on Nature-based solutions. The most relevant Nature-based solutions identified are floodplains, vegetated buffer zones, and riparian buffer zones, as they offer the most benefits for mitigating mining-related impacts. Floodplain restoration helps counteract land degradation and loss of biodiversity from dry sand mining by re-establishing natural hydrological connections and improves groundwater recharge. Vegetated buffer zones capture dust from sand mines, which helps protect nearby communities from air and sound pollution and any resulting negative health impacts. Riparian buffer zones filter pollutants and enhance biodiversity. In addition, all these solutions help reduce erosion. Although erosion in the delta could not be linked to sand extraction activities, the scale is significant and stakeholder interviews show that mitigation strategies are called for. Further, a further increase of dredging activities in the future is likely. Hence, future dredging-induced erosion is possible. As a structural solution to dredging, a sheet pile was deemed most appropriate. A final design, with depth 10.2 meters below the river bed and profile AZ24 700, is structurally sound and can thus help reduce erosion.

%Antwoord hoofdvraag

\section{Recommendations}
As mentioned in the previous chapter, this report should be read as an investigation of current circumstances. However, future research could adopt a broader scope by incorporating both historical and future-oriented data. Long-term monitoring of extraction rates, sediment transport, and hydrodynamic conditions would allow for a more robust understanding of variability and trends in the sediment balance. More continuous AIS data, data on sediment concentrations and hydrodynamic data are necessary for such an analysis. Furthermore, scenario-based modeling, with possible increases or decreases in fracking activity, would make it possible to predict how future developments can influence the socioeconomic effects of sand mining. By extending the modelling approach to a morphological model in Delft3D, the `turning point' at which dredging volumes are big enough to trigger bank instability and large-scale erosion, could be determined. In addition, tidal effects should be incorporated into the model, considering the significant influence of the Río de la Plata on the hydrodynamics in the Paraná Guazú.

Subsequent studies can also expand on the geographical scope of this analysis to capture the broader dynamics of sand extraction across Argentina. While this study focused on a limited area within the Lower Paraná Delta, similar activities occur throughout the rest of the delta and the country, under different conditions. A comprehensive nation-wide assessment would make it possible to determine whether the findings for the delta are representative. Moreover, this can help link the environmental consequences of sand mining to the impacts of its end uses, particularly fracking. A life-cycle or supply chain analysis would allow for tracing sand from extraction to consumption, thereby quantifying the total socioeconomic and environmental footprint of the sector on Argentina.

Building on this, future research should also aim to inform policy decisions through a broader, quantitative, cost–benefit analysis of sand extraction and its purposes. Such analyses could evaluate not only the direct profits from mining and related industries, but also the long-term costs associated with erosion, biodiversity loss, water degradation, and community impacts. Incorporating this with a nation-wide analysis of the footprint of the sand mining sector would allow for evidence-based decision-making and help put in place more sustainable extraction policies.