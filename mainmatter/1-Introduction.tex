\chapter{Introduction}
\label{chapter:introduction}

\section{Project Description}

Sand is one of the most important and most extracted natural resources globally. It is necessary for the production of everyday materials such as electronics, ceramics and glass as well as other infrastructural elements like concrete. For this reason, the global demand for sand has been rising linearly with the increase of population, explaining why it is mined in large quantities throughout the world \autocite{wwfRisingDemandSand}. 
In Argentina, the Lower Paraná delta is one of these locations that has become a critical spot for sand mining. Here, dredging companies extract  sand from the riverbed and dry sand mining companies extract it from the land. Historically, the sand was used mainly for construction but in recent years, hydraulic fracking has created a new source of demand.
While the extraction of sand can be beneficial for the economical development of the area, the impacts of the sand mining on the riverbed, the biodiversity and the surrounding delta remain insufficiently well understood to draw conclusions. 

This project aims to find out what the scale is of sand mining in the Lower Paraná delta, as well as establishing its diverse effects. By doing so, the study will reach its final goal: find solutions to mitigate the negative outcomes and find a balance with the local communities, current ecosystem and the sand demand. 

\begin{figure}[H]
    \centering    \includegraphics[width=1\linewidth]{figures/ch2/study area.png}
    \caption{Study area}
    \label{fig:study area}
\end{figure}

The study area focuses on a critical location of the Lower Paraná river as well as the surrounding delta, and includes a 15 kilometer long section of the Río Ibicuy. This river section later bifurcates into the Paraná to form the Paraná-Guazú at km 232.0. The analyzed river stretch extends roughly 70 kilometers, from Puerto Ibicuy to Puerto Guazú, near the Ruta Nacional 12, see Figure \ref{fig:study area} which provides an overview of the study area. Given the dredging activities, the project illustrates the presence of ports within the study area, since they serve as key points for the transport of the extracted sand. The study will be based on the study of wet but also dry sand mining.

\section{Problem Statement}

International case studies like the Mekong delta have shown the potential threat of large quantities of sand mining. The negative effects of significant amounts of extracted sand and aggregates in this case can be illustrated by highly irregular bed changes over the past 10 year comparison period, stressing on an increase of mean depth of 1.3 m due to bed material losses \autocite{brunierRecentMorphologicalChanges2014}. Concluding from the Mekong case study is that sand mining is the main factor contributing to the bank erosion in the delta as well as erosion in the delta shoreline. Furthermore, it can also be linked to increased morphological changes that will later impact the delta, especially in terms of sediment supply and coastal erosion.

This example is backed up by a study on environmental impacts on the river due to sand mining. It states that if sediment is removed faster than it can be replenished by natural transport processes, the system may experience an imbalance. The latter is expressed through channel deepening, bank erosion, and changes in flow patterns \autocite{rentierEnvironmentalImpactsRiver2022}. All of these changes can have considerable consequences for navigation, infrastructure stability, aquatic habitats, and flood risk. 

Considering the hypothesis that there may be an increasing trend of wet sand mining, the hydrodynamic regime of the river as well as the delta morphology may be at risk. Extracting large amounts of sand from the Lower Parana delta can thus affect the river's natural sediment balance. Therefore, to fully grasp the extent of these impacts, it is necessary to quantify the amounts extracted, and then compare the extraction rates with the river’s natural sediment transport capacity. Without a clear assessment of the situation, the long-term sustainability of the delta and the communities and industries dependent on it remains uncertain.

\section{Objectives}

% \subsection{Project strategy}
% \subsection{Methods}
% \subsection{Multidisciplinary approach}
% \subsection{Methodology}

One might agree with the studies cited that continued sand mining can cause irreversible damage to the ecological and socioeconomic environment. Therefore, it is suggested to study the evolution of sand extraction in the Lower delta area and its potential projections and implications. Specifically, an answer to the following research question is sought:

\textit{What are the effects of sand extraction in the Lower Paraná delta and how can these be managed to secure a sustainable future?}

In addition, a number of sub-questions are formulated to guide the reasoning of this report. The project aims to integrate perspectives from multiple disciplines and present the results in a coherent manner.

\begin{itemize} 
    \item How much sand is extracted in the Lower Paraná delta and what are its main uses?
    \item How can the sediment balance of the Paraná Guazú river be established and quantified?
    \item What are the characteristics of river flow patterns and hydrodynamics in the study area?
    \item What are the erosion processes in the Paraná Guazú river and how are they related to dredging activities?
    \item What drives demand for sand from the Lower Paraná delta?
    \item What are the effects of sand extraction in the Lower Paraná delta?
    \item Which mitigation strategies can be proposed to reduce negative impacts?
\end{itemize}

\section{Multidisciplinary approach}
The aim of this report is to present a comprehensive assessment of the sustainability of sand mining in the Paraná River.
To address the complex study, this report will adopt a multidisciplinary approach, integrating hydraulic, geotechnical, and structural engineering.
Accordingly, the impacts of sand mining are examined from several perspectives, reflecting the specialisation of the team members:

\begin{itemize}
    \item \textbf{Hydraulic}: 
    Processing available hydrodynamic data.
    Assessment of the sediment balance in the Paraná Guazú River.
    Understanding the river flow in the region of interest using hydrodynamic model Delft3D. Furthermore, Evaluate the tide, wave interactions and effects on the bank erosion and delta. 
    
    \item \textbf{Geotechnical}: 
    Investigate the dry sand mining activities and the geological/geotechnical characteristics that drive the demand for sand from the region. Finally, identify possible Nature-based mitigation measures for the dry sand mining practices.
    \item \textbf{Structural}: Consider different structural solutions for bank erosion and come up with a 2D-design for the most feasible one.
\end{itemize}

\section{Report Outline}
\label{section: report outline}
This section describes the structure of the report. Chapter 2 provides the background study, outlining the context of the problem and supporting later findings with theoretical foundations. Chapter 3 explains the methodology, detailing the data collection process, measurement techniques, and the multidisciplinary character of the project by linking each discipline to a corresponding part of the research. Chapter 4 presents a stakeholder analysis, identifying the key parties involved in sand extraction activities and the interview results. Chapter 5 focuses on the sand extraction in the designated area, linking the wet and dry sand mining to the project. Chapter 6 recalls all the hydrodynamic theory and data, which is later used in Chapter 7 for the Delft3D model. Building on these findings, Chapter 8 proposes mitigation strategies to address the identified impacts. Finally, Chapters 9 and 10 conclude the report with a discussion and conclusion, respectively.


