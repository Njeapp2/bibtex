\chapter{Introduction}
\label{chapter:introduction}

Chapter 1.1 presents the context of the project, outlining the reasons behind its initiation. Chapter 1.2 then provides a detailed description of the problem at hand. Finally, Chapter 1.3 discusses an outline of its primary objectives, specifying what the project seeks to accomplish.

\section{Project Description}

Sand is one of the most important and most extracted natural resources. It is necessary for the production of everyday materials such as glass, concrete, ceramics, electronics as well as other infrastructure processes. For this reason, the global demand for sand has been increasing linearly with the change of demography, explaining why it is mined in large quantities throughout the world. 
In Argentina, the Lower Paraná Delta is one of these locations that has become a critical spot for sand mining. In this very location dredging companies extract large amounts of sand from the riverbed. The extracted sand is used mainly for construction but in recent years its demand has surged due to the fairly recent process of hydraulic fracking in the area.

While the extraction of sand is needed for the economical development of the area, the impacts of the sand mining for the riverbed, the biodiversity and the surrounding Delta remain insufficiently well understood to draw conclusions. 

This project aims to find out what the scale is of sand mining in the Lower Paraná Delta, as well as establishing its diverse effects. By doing so, the study will reach its final goal: find solutions to mitigate the negative outcomes and find a balance with the local communities, current ecosystem and the sand demand. 



\begin{figure}[H]
    \centering    \includegraphics[width=0.70\linewidth]{figures/ch2/Paraná.png}
    \caption{Study area}
    \label{fig:study area}
\end{figure}
\label{Figure 1.1}

The study area focuses on a critial location of the Lower Paraná river, and includes a 15 kilometer long section of the Río Ibicuy, which later bifurcates into the Paraná to form the Paraná-Guazú at km 232.0. The analyzed river stretch extends roughly 70 kilometers, from Puerto Ibicuy to Puerto Guazú, near the Ruta Nacional 12, see Figure \ref{fig:study area} which provides an overview of the study area. Given the dredging activities, the project illustrates the presence of ports within the study area, since they serve as key points for the transport of the extracted sand.



\section{Problem Statement}

On the other hand, international case studies like the Mekong Delta have shown the negative effects of enormous amounts of extracted sand \autocite{brunierRecentMorphologicalChanges2014}. 


The large amounts of sand extracted from the Lower Parana Delta affect the river's natural sediment balance. Therefore, to fully grasp the extent of these impacts, it is highly necessary to quantify the amounts extracted, and then compare the extraction rates with the river’s natural sediment transport capacity.

Considering the increasing trends of sand mining, the hydrodynamic regime of the river may be at risk. If sediment is removed faster than it can be replenished by natural transport processes, the system may experience channel deepening, bank erosion, and changes in flow patterns. These changes can have considerable consequences for navigation, infrastructure stability, aquatic habitats, and flood risk. 

Without a clear assessment of the sediment balance, the long-term sustainability of the Delta and the communities and industries dependent on it remains uncertain.

\section{Objectives}

% \subsection{Project strategy}
% \subsection{Methods}
% \subsection{Multidisciplinary approach}
% \subsection{Methodology}

It is now well understood that continued sand mining can cause irreversible damages to the ecological and socioeconomic environment. Therefore, it is suggested to study the evolution of sand extraction in the Paraná Guazú river and its potential projections and implications. Specifically, an answer is sought to the following research question:

\textit{What are the effects of sand extraction in the Lower Paraná
Delta and how can these be managed to secure a sustainable future?}

In addition, a number of sub-questions are formulated to guide the reasoning of this report. The project aims to integrate perspectives from multiple disciplines and present the results in a coherent manner.

\begin{itemize} 
    \item What are the effects of sand extraction in the Paraná Guazú River?
    \item How can the sediment balance of the study area be established and quantified?
    \item In what ways does sand extraction alter river flow patterns and hydrodynamics?
    \item How do these changes affect existing infrastructure along the river?
    \item What is the impact of extraction on riverbank stability and erosion processes?
    \item What are the implications on the Paraná delta?
    \item Which mitigation strategies can be proposed to reduce negative impacts?
\end{itemize}

The expected outcomes for the subquestions can be found in the section \ref{section: report outline}.

\section{Multidisciplinary approach}

The aim of this report is to present a comprehensive assessment of the sustainability of sand mining in the Paraná River. To capture the full scope of the issue, a multidisciplinary approach is adopted. Accordingly, the impacts of sand mining are examined from several perspectives, reflecting the expertise of the authors:

\begin{itemize}
    \item \textbf{Hydraulic}: assessment of the sediment balance and future river flow projections.
    \item \textbf{Geotechnical}: evaluation of risks to riverbank stability and identification of possible mitigation measures.
    \item \textbf{Structural}: analysis of potential effects on infrastructure, along with consideration of structural measures such as quay walls or stone revetments.
\end{itemize}

\section{Report Outline}
\label{section: report outline}
This section describes the structure of the report. Chapter 2 provides the background study, outlining the context of the problem and supporting later findings with theoretical foundations. Chapter 3 presents a stakeholder analysis, identifying the key parties involved in sand extraction activities. Chapter 4 explains the methodology, detailing the data collection process, measurement techniques, and the multidisciplinary character of the project by linking each discipline to a corresponding part of the research. Chapter 5 focuses on data analysis and processing, while Chapter 6 presents the results derived from this analysis. Building on these findings, Chapter 7 proposes mitigation strategies to address the identified impacts. Finally, Chapters 8 and 9 conclude the report with a discussion and conclusion, respectively.











