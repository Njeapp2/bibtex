\chapter*{Summary}
\addcontentsline{toc}{chapter}{Summary}
Sand is one of the most extracted natural resources worldwide, and demand continues to rise as population growth and infrastructure development intensify. In Argentina, the Lower Paraná Delta has become a key source of sand for both construction and hydraulic fracturing (“fracking”) activities. While sand mining generates economic benefits and supports industrial growth, its environmental and socioeconomic impacts on the delta remain poorly understood. This study aims to determine the scale of sand extraction in the Lower Paraná Delta and to assess its morphological and socioeconomic effects on the river system and surrounding land.

The research applied a multidisciplinary approach combining hydraulic, geotechnical, and structural engineering perspectives. Quantitative analyses were based on field measurements, sediment sampling, and hydrodynamic modelling with Delft3D. Additionally, stakeholder interviews and data from the Automatic Identification System (AIS) were used to assess extraction volumes and local perceptions. This combination allowed for a comparative evaluation of river (wet) and land-based (dry) sand mining and their respective effects on the delta.

Results show that while river sand extraction remains relatively stable at around 587,520 tons per year, dry sand mining has increased sharply to approximately 2.3 million tons in 2025, mainly driven by the demand for fracking sand. The established sediment balance of the Paraná Guazú River reveals a net negative flux of roughly 15,400 tons per day, suggesting sediment depletion. Hydrodynamic analyses indicate weak stage–discharge correlations and strong influence of meteorological tides. Erosion rates observed in the study area range between 3–7 m per year, with natural processes such as river meandering and flood-induced bank instability identified as the dominant drivers. Socioeconomic impacts are more evident for dry mining, including groundwater overuse, road damage, and habitat loss.