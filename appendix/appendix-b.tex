\chapter{Reflection and Task Division}
\label{chapter:Reflection and Task Division}

\section{Reflection}
Throughout our MDP project, our group maintained a collaborative approach to ensure efficiency and good communication. To optimize our workflow, we divided the team into two groups of three: one attended INA on Mondays, while the other went on Wednesdays. This allowed us to maintain a consistent presence at INA while ensuring all members had equal access to resources and stakeholders. On Tuesdays, the entire group worked at the Boskalis office, creating a productive team alignment since we had access to a blackboard. The remaining days were dedicated to remote work, where we operated as a unified group from home, adhering to a standard internship schedule from 9 to 17, in the living room on the dinner table.


Communication with our supervisors primarily took place during our office days at INA. Outside these days, we relied on digital platforms such as Teams, WhatsApp, and email to stay connected and address any questions or updates. This hybrid approach ensured continuous guidance and support, regardless of our physical location.


Deadlines were managed proactively through a planning process, or spontaneous meetings outside working hours which we developed as a group at the outset of the project. Each milestone and task was discussed collectively, allowing us to give out responsibilities effectively and ensure everyone was aligned with the project’s main question. This approach not only kept us on track but also encouraged open communication and constructive debates during group work sessions.


Overall, the combination of in-person and remote collaboration, clear communication channels, and a well-structured planning process contributed to a productive and cohesive team dynamic. Our ability to adapt to different working environments—whether at INA, Boskalis, or remotely—strengthened our multidisciplinary integration and ensured the successful fulfillment of project requirements.

\section{Task Division}

Throughout the ten weeks of the project every student of the MDP385 Group has actively contributed to the project. 
The contributions have been allocated to one of several students per Chapter in the Report. This can be seen in the Table below.


\begin{table}[htb]
    \setlength\extrarowheight{4pt}
    \centering
    \caption{Distribution of the workload}
    \label{tab:taskdivision}
    \begin{tabularx}{\textwidth}{lXX}
        \toprule
        & Task & Student Name(s) \\
        \midrule
        & Summary & All \\
        Chapter 1 & Introduction & All \\
        \midrule
        Chapter 2 & Background Study & \\
        2.1 & Argentina’s waterways & Victor \\
        2.2 & Classification of Rio Paraná & Victor \\
        2.3 & Origin of sediment content in Paraná Guazú & Jasper \\
        2.4 & Mining of the Sand and Types of Dredging in a River & Niek \\
        2.5 & The effects of river sand mining & Niek \\
        \midrule
        Chapter 3 & Methodology & \\
        3.1 & Data collection & Jasper Stefan\\
        3.2 & Field study & Victor Laurens \\
        3.3 & Setting up the Sediment Balance & Stefan \\
        3.4 & Bank Erosion & Victor \\
        3.5 & Multidisciplinary approach &  Mike \\
        \midrule
        Chapter 4 & Stakeholders & \\
        4.1 & Stakeholder analysis & Mike Niek\\
        4.2 & Interview results & Niek \\
        4.3 & Updated stakeholder analysis & Mike \\
        \midrule
        Chapter 5 & Sand extraction & \\
        5.1 & Dredging activities & Laurens \\
        5.2 & Dry sand mining & Niek Laurens\\
        \midrule
        Chapter 6 & Hydrodynamical and sedimentary analysis & \\
        6.1 & Effect of tides and waves on the water level & Victor\\
        6.2 & Hydrodynamic data & Jasper Stefan\\
        6.3 & Sediment transport & Stefan \\
        6.4 & Field work measurements & Laurens\\
        6.5 & Hydrodynamic effects on the River Banks & Victor \\
        \bottomrule
    \end{tabularx}
\end{table}

\begin{table}[htb]
    \setlength\extrarowheight{4pt}
    \centering
    \caption{Distribution of the workload}
    \label{tab:taskdivision}
    \begin{tabularx}{\textwidth}{lXX}
        \toprule
                Chapter 7 & Delft3D Model & Jasper Stefan\\
        \midrule
        Chapter 8 & Mitigation Strategies & \\
        8.1 & Nature-based Solutions for bank erosion & Laurens \\
        8.2 & Structural Solutions for bank erosion & Mike \\
        8.3 & Nature-based Solutions for dry sand mining & Niek \\
        \midrule
        Chapter 9 & Discussion & Niek \\
        \midrule
        Chapter 10 & Conclusion and recommendations & All \\
        \midrule
        Appendix A & Codes?? & Mike \\
        Appendix B & Reflection and Task Division & Victor \\
        Appendix C & Safety Assessment & Laurens \\
        Appendix D & Unprocessed interview results & Niek \\
        Appendix E & Laboratory Data & Victor Laurens \\
        Appendix F & Hydrodynamic Satellite Data & Victor \\
        Appendix G & ADCP Results & Stefan Jasper\\
        \midrule
        & Document Design and Layout & All \\
        \bottomrule
    \end{tabularx}
\end{table}

\section{Fieldwork Tasks and Planning}

For the fieldwork, different tasks were assigned, which were not included in the task division table. 
Therefore, the pdf document used during the trip has been translated and put down below.
Note that repeated info such as list of stakeholders and interview questions have not been included.

\subsubsection{Wednesday (DAY 1)}
8:00 Departure from BA (Group 1 and Group 2)
Group 1: Stefan, Victor, Jasper
Group 2: Niek, Laurens, Mike
Group 1: Boat Group
10:00 Arrive at Recreo Keidel, drop off and set up camping gear, and pay (250,000 ARS).
11:00 Meet the fisherman’s boat with Martin (give money for boat).
11:30 Prepare the boat for Thursday (confirm time, location, and details).
12:30 Prepare equipment and ask for explanations to avoid losing time.
Lunch
Drive to stakeholders (45 min) east of RN12 bridge.
14:30 Camping La Torre
15:30 Yacht Club Guazú
Drive back to camping (30 min)
16:30 Arrive back at camp or similar.
Group 2: Stakeholders (car with marina)
Proposal: Drive directly to Ibicuy to conduct the first interviews and avoid traveling back and forth from the campsite (2.5-hour drive from 9 de Julio Avenue).
Arrive around 10:30–11:00.
Morning interviews if possible, otherwise, schedule after appointments:

Camping Los Abuelos
Camping Boca Del Pavón
Club de Pescadores Olivos (Ibicuy branch)
If time allows, visit the northern end of the area:
Camping El Islerito (above Ibicuy port, quite remote)
Lunch
13:00 Meeting with Ibicuy Port Manager
15:00 Meeting with Ibicuy Mayor
Drive back to camping (1 hour 15 min)
17:00 Arrive back at camp or similar. If earlier, conduct additional interviews around Puerto Guazú.
Ask Marina to translate live if recording is not allowed.


\subsubsection{Thursday (DAY 2)}
Group 1: Stefan, Jasper, Niek
Group 2: Laurens, Mike, Victor
Group 1: Boat Group
Start as early as possible to avoid returning late.
8:00–9:00 Board the boat.
Measurements:

Cross Sections 1, 2, 3 around the dredging point
Discharge measurement with SonTek M9
Flow velocity with radar
Bed load with metal digging tool
Sediment concentration with bottles
Soil sample with excavator
Record notes and trajectory in Locus Map for later GIS import and report inclusion.
Lunch on the boat.
Switch tasks during lunch if possible.
Return to camping by 18:00.

Group 2: Stakeholders
Stay near Guazú for stakeholders; possibly depart at 8:30 (50 min drive to Puerto Guazú).
9:30 Arrive in Puerto Guazú.

9:30 Camping Iponá Guazú
10:00 El Molino Camping
10:30 Port Guazú (Do we have a meeting scheduled, or should we improvise?)
11:00–12:00 Visit Arenera del Guazú to observe small dredging vessels.
Lunch
Drive to Constanza (30 min).
At two small harbors (Constanza and another):
14:00 Camping El Amanecer (near two small harbors)
15:00 Camping La Blanqueada
15:30 Oasis Guazú
Drive back to camping (1 hour).


\subsubsection{Friday (DAY 3)}
Group 1: Laurens, Mike, Victor
Group 2: Jasper, Stefan, Niek
Group 1: Boat Group
8:00–9:00 Board the boat.
1–1:30 Travel to Cross Section 4 in Ibicuy.
Measurements:

Cross Section 4 in Port Ibicuy
Discharge measurement with SonTek M9
Flow velocity with radar
Bed load with metal digging tool
Sediment concentration with bottles
Soil sample with excavator
Record notes and trajectory in Locus Map for later GIS import and report inclusion.
Lunch on the boat.
1–1:30 Travel to Cross Section 4 in Ibicuy.
16:30 Return to camp (estimated by Martin).

Group 2: Stakeholders
9:00 Pack up camping gear and check out.
Reserve day for stakeholders or redistribute tasks, e.g., visit the east side of Puerto Guazú if Group 1 couldn’t on Wednesday. Alternatively, summarize stakeholder findings.
Or:
Drive 45 min east of RN12 bridge.
14:30 Camping La Torre
15:30 Yacht Club Guazú
Drive back to camping (30 min).
Groups 1 and 2:
Drive back to Buenos Aires together, dropping off at strategic locations for faster travel home.
1.5–2 hours drive back to BA.
19:00 Arrive home or similar.