
\chapter{Safety assessment}

This appendix presents the risk assessment associated with the procedures that will be conducted in the fieldwork as part of the multiple discipline project titled "Delft University of Technology Sediment Balance in a Sector of the Paraná Guazú River"

\section{Risk Assessment: Fieldwork}

\subsection{Hazard Identification}
This fieldwork includes a lot of work around and on the water. Working around water involves several hazards that may affect the safety and health of people in the group. The most extreme risk is the risk of drowning, for example with falls of the boat, unstable banks or strong water currents. Weather conditions can create additional risks such heat stress due to the sun. Contact with water may also result in biological hazards, such as exposure to bacteria. People may also take caution when using mechanical equipment. Finally there are also risks from wildlife (shells, bites and insects). 

\subsection{Risk Assessment}
Risks vary in likelihood and consequence. The risk of drowning is assessed as low in likelihood but is very high in consequence, this makes it a critical concern. The risk due to weather conditions is in likelihood quite high and the consequence is moderate. This makes it a risk which should be mitigated. Biological hazards are quite low in likelihood but are of high consequence, this is why it's important that every staff member is aware of this risk. Mechanical equipment hazard is in our case low in likelihood and moderate in consequence. 


\subsection{Control Measures}
To minimize risks it is important to use some personal protective equipment (PPE) such as life jackets, good footwear which has enough grip, gloves (depending on the work). Every group member must me aware of the risks, that's why there is a safety breefing before operating the fieldwork. Hygiene measures helps reduce biological risks, that's why it's advisable to wash your hands regularly. If there is any form of emergency it's important that everyone is reachable on short notice. That is why everyone should always carry his phone around. 