
\chapter{Safety assessment}

This appendix presents the risk assessment associated with the procedures that will be conducted in the fieldwork as part of the multiple discipline project titled "Delft University of Technology Sediment Balance in a Sector of the Paraná Guazú River"

\section{Risk Assessment: Fieldwork}

\subsection{Hazard Identification}
This fieldwork involves extensive activities both on and around the water, which inherently present a variety of safety and health hazards. Working in such environments requires continuous awareness and precautionary measures to protect all members of the research team.
The most severe hazard associated with water-based fieldwork is the risk of drowning. This can occur as a result of falling from boats, working on unstable or slippery riverbanks, or being caught in strong currents. Proper use of life jackets, safe boarding procedures, and clear communication within the group are therefore essential preventive measures.

In addition to the immediate risks of drowning, environmental and weather-related factors can also impact safety. High temperatures and prolonged exposure to the sun may cause heat stress, dehydration, or sunburn, while sudden changes in weather—such as heavy rainfall, thunderstorms, or strong winds—can rapidly increase danger on the water. Adequate protective clothing, hydration, and weather monitoring should be part of the fieldwork routine.

Contact with surface water may also expose researchers to biological hazards. Natural water bodies can contain bacteria, parasites, or other microorganisms that cause skin infections or gastrointestinal illness. Wearing waterproof gloves, avoiding direct contact with open wounds, and practicing proper hygiene (e.g., hand washing or use of disinfectants after fieldwork) are effective ways to reduce these risks.
Another aspect to consider is the safe handling of mechanical and sampling equipment. Working with pumps, sieves, augers, or motorized boats requires attention to mechanical hazards such as entanglement, cuts, or equipment malfunction. Ensuring that all equipment is well maintained and operated only by trained individuals minimizes these dangers.

Finally, the natural environment itself may present additional threats from wildlife. These can range from sharp shells or stinging organisms in the water, to insect bites, snakes, or other animals encountered near the riverbank. Using appropriate footwear, insect repellent, and maintaining awareness of the surroundings can help prevent injuries or allergic reactions.


\subsection{Risk Assessment}
The risks associated with this fieldwork vary both in their likelihood of occurrence and in the severity of their potential consequences. Understanding this balance is essential for prioritizing safety measures and ensuring that all critical hazards receive appropriate attention and control.

The risk of drowning, although assessed as having a low likelihood under normal operating conditions, carries extremely severe consequences should it occur. Because of the potentially fatal outcome, this hazard remains a critical concern and must always be treated with the highest level of precaution. Preventive actions such as the mandatory use of life jackets, maintaining clear safety protocols during boat operations, and ensuring all participants are trained in emergency procedures are therefore non-negotiable components of field safety.

Risks arising from weather conditions are more likely to occur and can vary throughout the day. The likelihood of exposure to heat, sun, or sudden changes in weather is considered relatively high, while the potential consequences—ranging from mild heat stress to temporary work interruptions—are moderate. Nevertheless, such risks should be actively mitigated through measures including weather monitoring, adequate rest and hydration, the use of sun protection, and flexible planning to avoid dangerous conditions.

Biological hazards, such as exposure to bacteria or other pathogens present in the water, are generally considered to have a low likelihood of occurrence if proper hygiene and protective measures are followed. However, the consequences of such exposure can be significant, including illness or infection. For this reason, it is vital that all team members remain aware of these hazards and adhere strictly to personal protection and sanitation guidelines, such as using gloves, avoiding contact with open wounds, and washing hands thoroughly after field activities.

Finally, hazards related to mechanical equipment—such as cuts, entanglement, or mechanical malfunction—are assessed as having a low likelihood in this specific fieldwork setting. Their potential consequences are moderate, primarily involving minor injuries or temporary disruption of operations. Routine maintenance, proper training, and adherence to safe operating procedures are sufficient to keep this risk at an acceptable level.


\subsection{Control Measures}
To minimize potential risks during fieldwork, it is essential that all participants make consistent and appropriate use of personal protective equipment (PPE). Key items include life jackets when working on or near the water, sturdy footwear with sufficient grip to prevent slipping on wet or uneven surfaces, and protective gloves when handling tools, equipment, or biological materials. The type of gloves may vary depending on the specific task—ranging from waterproof gloves for wet environments to cut-resistant gloves when working with sharp instruments.

Equally important is ensuring that every team member is fully aware of the hazards present in the field environment. To achieve this, a comprehensive safety briefing must be conducted before any field activities begin. During this briefing, all risks, safety procedures, and emergency response protocols should be clearly explained, allowing participants to understand both their individual responsibilities and the collective safety measures of the group.

Maintaining proper hygiene also plays a crucial role in minimizing biological risks. Regular handwashing, particularly before eating or after contact with river water, significantly reduces the likelihood of bacterial or parasitic infections. The availability of clean water, disinfectant wipes, or alcohol-based sanitizers should therefore be ensured at all times.

In the event of an emergency, effective communication is vital. All team members must be reachable at short notice to enable rapid coordination and response. For this reason, everyone should carry a charged mobile phone at all times. A clear communication plan—including designated contact persons and emergency numbers—should be established before fieldwork begins.

By combining the proper use of PPE, strong situational awareness, good hygiene practices, and reliable communication systems, the risks associated with water-based fieldwork can be significantly reduced, ensuring a safe and efficient working environment for all participants.